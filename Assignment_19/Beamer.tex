\documentclass[journal,12pt,twocolumn]{IEEEtran}
%
\usepackage{setspace}
\usepackage{gensymb}
\usepackage{siunitx}
\usepackage{tkz-euclide} 
\usepackage{textcomp}
\usepackage{standalone}
\usetikzlibrary{calc}

%\doublespacing
\singlespacing

%\usepackage{graphicx}
%\usepackage{amssymb}
%\usepackage{relsize}
\usepackage[cmex10]{amsmath}
%\usepackage{amsthm}
%\interdisplaylinepenalty=2500
%\savesymbol{iint}
%\usepackage{txfonts}
%\restoresymbol{TXF}{iint}
%\usepackage{wasysym}
\usepackage{amsthm}
%\usepackage{iithtlc}
\usepackage{mathrsfs}
\usepackage{txfonts}
\usepackage{stfloats}
\usepackage{bm}
\usepackage{cite}
\usepackage{cases}
\usepackage{subfig}
%\usepackage{xtab}
\usepackage{longtable}
\usepackage{multirow}
%\usepackage{algorithm}
%\usepackage{algpseudocode}
\usepackage{enumitem}
\usepackage{mathtools}
\usepackage{steinmetz}
\usepackage{tikz}
\usepackage{circuitikz}
\usepackage{verbatim}
\usepackage{tfrupee}
\usepackage[breaklinks=true]{hyperref}
%\usepackage{stmaryrd}
\usepackage{tkz-euclide} % loads  TikZ and tkz-base
%\usetkzobj{all}
\usetikzlibrary{calc,math}
\usepackage{listings}
    \usepackage{color}                                            %%
    \usepackage{array}                                            %%
    \usepackage{longtable}                                        %%
    \usepackage{calc}                                             %%
    \usepackage{multirow}                                         %%
    \usepackage{hhline}                                           %%
    \usepackage{ifthen}                                           %%
  %optionally (for landscape tables embedded in another document): %%
    \usepackage{lscape}     
\usepackage{multicol}
\usepackage{chngcntr}
\usepackage{amsmath}
\usepackage{cleveref}
%\usepackage{enumerate}

%\usepackage{wasysym}
%\newcounter{MYtempeqncnt}
\DeclareMathOperator*{\Res}{Res}
%\renewcommand{\baselinestretch}{2}
\renewcommand\thesection{\arabic{section}}
\renewcommand\thesubsection{\thesection.\arabic{subsection}}
\renewcommand\thesubsubsection{\thesubsection.\arabic{subsubsection}}

\renewcommand\thesectiondis{\arabic{section}}
\renewcommand\thesubsectiondis{\thesectiondis.\arabic{subsection}}
\renewcommand\thesubsubsectiondis{\thesubsectiondis.\arabic{subsubsection}}

% correct bad hyphenation here
\hyphenation{op-tical net-works semi-conduc-tor}
\def\inputGnumericTable{}                                 %%

\lstset{
%language=C,
frame=single, 
breaklines=true,
columns=fullflexible
}
%\lstset{
%language=tex,
%frame=single, 
%breaklines=true
%}
\usepackage{graphicx}
\usepackage{pgfplots}

\begin{document}
%


\newtheorem{theorem}{Theorem}[section]
\newtheorem{problem}{Problem}
\newtheorem{proposition}{Proposition}[section]
\newtheorem{lemma}{Lemma}[section]
\newtheorem{corollary}[theorem]{Corollary}
\newtheorem{example}{Example}[section]
\newtheorem{definition}[problem]{Definition}
%\newtheorem{thm}{Theorem}[section] 
%\newtheorem{defn}[thm]{Definition}
%\newtheorem{algorithm}{Algorithm}[section]
%\newtheorem{cor}{Corollary}
\newcommand{\BEQA}{\begin{eqnarray}}
\newcommand{\EEQA}{\end{eqnarray}}
\newcommand{\define}{\stackrel{\triangle}{=}}
\bibliographystyle{IEEEtran}
%\bibliographystyle{ieeetr}
\providecommand{\mbf}{\mathbf}
\providecommand{\pr}[1]{\ensuremath{\Pr\left(#1\right)}}
\providecommand{\qfunc}[1]{\ensuremath{Q\left(#1\right)}}
\providecommand{\sbrak}[1]{\ensuremath{{}\left[#1\right]}}
\providecommand{\lsbrak}[1]{\ensuremath{{}\left[#1\right.}}
\providecommand{\rsbrak}[1]{\ensuremath{{}\left.#1\right]}}
\providecommand{\brak}[1]{\ensuremath{\left(#1\right)}}
\providecommand{\lbrak}[1]{\ensuremath{\left(#1\right.}}
\providecommand{\rbrak}[1]{\ensuremath{\left.#1\right)}}
\providecommand{\cbrak}[1]{\ensuremath{\left\{#1\right\}}}
\providecommand{\lcbrak}[1]{\ensuremath{\left\{#1\right.}}
\providecommand{\rcbrak}[1]{\ensuremath{\left.#1\right\}}}
\theoremstyle{remark}
\newtheorem{rem}{Remark}
\newcommand{\sgn}{\mathop{\mathrm{sgn}}}



\providecommand{\abs}[1]{\left\vert#1\right\vert}
\providecommand{\res}[1]{\Res\displaylimits_{#1}} 
\providecommand{\norm}[1]{\left\lVert#1\right\rVert}
%\providecommand{\norm}[1]{\lVert#1\rVert}
\providecommand{\mtx}[1]{\mathbf{#1}}
\providecommand{\mean}[1]{E\left[ #1 \right]}
\providecommand{\fourier}{\overset{\mathcal{F}}{ \rightleftharpoons}}
%\providecommand{\hilbert}{\overset{\mathcal{H}}{ \rightleftharpoons}}
\providecommand{\system}{\overset{\mathcal{H}}{ \longleftrightarrow}}
	%\newcommand{\solution}[2]{\textbf{Solution:}{#1}}
\newcommand{\solution}{\noindent \textbf{Solution: }}
\newcommand{\cosec}{\,\text{cosec}\,}
\providecommand{\dec}[2]{\ensuremath{\overset{#1}{\underset{#2}{\gtrless}}}}
\newcommand{\myvec}[1]{\ensuremath{\begin{pmatrix}#1\end{pmatrix}}}
\newcommand{\mydet}[1]{\ensuremath{\begin{vmatrix}#1\end{vmatrix}}}
%\numberwithin{equation}{section}
\numberwithin{equation}{subsection}
%\numberwithin{problem}{section}
%\numberwithin{definition}{section}
\makeatletter
\@addtoreset{figure}{problem}
\makeatother
\let\StandardTheFigure\thefigure
\let\vec\mathbf
%\renewcommand{\thefigure}{\theproblem.\arabic{figure}}
\renewcommand{\thefigure}{\theproblem}
%\setlist[enumerate,1]{before=\renewcommand\theequation{\theenumi.\arabic{equation}}
%\counterwithin{equation}{enumi}
%\renewcommand{\theequation}{\arabic{subsection}.\arabic{equation}}
\def\putbox#1#2#3{\makebox[0in][l]{\makebox[#1][l]{}\raisebox{\baselineskip}[0in][0in]{\raisebox{#2}[0in][0in]{#3}}}}
     \def\rightbox#1{\makebox[0in][r]{#1}}
     \def\centbox#1{\makebox[0in]{#1}}
     \def\topbox#1{\raisebox{-\baselineskip}[0in][0in]{#1}}
     \def\midbox#1{\raisebox{-0.5\baselineskip}[0in][0in]{#1}}
\vspace{3cm}
\title{Matrix Theory (EE5609) Assignment 19}
\author{Arkadipta De\\MTech Artificial Intelligence\\AI20MTECH14002}

\maketitle
\newpage
%\tableofcontents
\bigskip
\renewcommand{\thefigure}{\theenumi}
\renewcommand{\thetable}{\theenumi}

\begin{abstract}
This document solves a problem on a functional. 
\end{abstract}
All the codes for the figure in this document can be found at
\begin{lstlisting}
https://github.com/Arko98/EE5609/blob/master/Assignment_19
\end{lstlisting}
\section{\textbf{Problem}}
Let $\mathbb{V}$ be the vector space of all $2 \times 2$ matrices over the field of real numbers, and let
\begin{align*}
\vec{B} &= \myvec{2&-2\\-1&1}
\end{align*}
Let $\mathbb{W}$ be the subspace of $\mathbb{V}$ consisting of all $\vec{A}$ such that $\vec{AB} = 0$. Let $f$ be a linear functional on $\mathbb{V}$ which is in the annihilator of $\mathbb{W}$. Suppose that $f(\vec{I}) = 0$ and $f(\vec{C}) = 3$, where $\vec{I}$ is the $2 \times 2$ identity matrix and
\begin{align*}
\vec{C} &= \myvec{0&0\\0&1}
\end{align*}
Find $f(\vec{B})$
\section{\textbf{Solution}}
Let,
\begin{align}
\vec{A} &= \myvec{a&b\\c&d} \quad{\text{$\forall$ $\vec{A} \in \mathbb{W}$}}
\end{align}
From $\vec{AB} = 0$ we have,
\begin{align}
\myvec{x&y\\z&w}\myvec{2&-2\\-1&1} = \myvec{0&0\\0&0}\label{eq2}
\end{align}
From \eqref{eq2} we get,
\begin{align}
y &= 2x\label{eqy}\\
w &= 2z\label{eqw}
\end{align}
Hence, using \eqref{eqy} and \eqref{eqw} we conclude that $\mathbb{W}$ consists of all the matrices of the following form,
\begin{align}
\vec{A} &= \myvec{x&2x\\z&2z} \quad{\text{$\forall$ $\vec{A} \in \mathbb{W}$}}\label{eqw}
\end{align}
Hence from \eqref{eqw} we get,
\begin{align}
A &\in \left\{a\myvec{1&2\\0&0}+b\myvec{0&0\\1&2}\right\} \quad{\text{$\forall$ $a,c \in \mathbb{R}$}}\\
\implies span(\mathbb{W}) &= \left\{\vec{e_1} = \myvec{1&2\\0&0}, \vec{e_2} = \myvec{0&0\\1&2}\right\}
\end{align}
Hence, $\vec{e_1}$ and $\vec{e_2}$ are basis of $\mathbb{W}$.
Since $f \in \mathbb{W}^0$, it follows that,
\begin{align}
f(\vec{e_1}) &= 0\label{v1}\\
f(\vec{e_2}) &= 0\label{v2}
\end{align}
Again given,
\begin{align}
f(\vec{I}) &= 0\label{v3}\\
f(\vec{C}) &= 0\label{v4}
\end{align}
From \eqref{v1}, \eqref{v2}, \eqref{v3} and \eqref{v4} we conclude that $\vec{I},\vec{C},\vec{e_1}, \vec{e_2}$ are linearly independent. We prove the statement as follows,
\begin{align}
c_1\myvec{1&0\\0&1}+c_2\myvec{0&0\\0&1}+c_3\myvec{1&2\\0&0}+c_4\myvec{0&0\\1&2} &= \myvec{0&0\\0&0}\\
\implies \myvec{c_1+c_3&2c_3\\c_4&c_1+c_2+2c_4} &= \myvec{0&0\\0&0}\\
\implies c_1 = c_2 = c_3 = c_4 &= 0\label{ind}
\end{align}
Since $dim(\mathbb{V}) = 2\times2 = 4$, hence the set $\left\{\vec{I},\vec{C},\vec{e_1}, \vec{e_2}\right\}$ forms a basis for the vector-space $\mathbb{V}$ which is the vector-space of all $2 \times 2$ matrices.\\
Since $\vec{B} \in \mathbb{V}$ then there exists $c_1,c_2,c_3,c_4$ such that,
\begin{align}
&c_1\vec{I}+c_2\vec{C}+c_3\vec{e_1}+c_4\vec{e_2} =\vec{B}\\
&c_1\myvec{1&0\\0&1}+c_2\myvec{0&0\\0&1}+c_3\myvec{1&2\\0&0}+c_4\myvec{0&0\\1&2}\\
&=\myvec{2&-2\\-1&1}\\
&\myvec{c_1+c_3&2c_3\\c_4&c_1+c_2+2c_4} = \myvec{2&-2\\-1&1}\label{sys}\\
\end{align}
From \eqref{sys} we get,
\begin{align}
c_1 &= 3\label{val1}\\
c_2 &= 0\label{val2}\\
c_3 &= -1\label{val3}\\
c_4 &= -1\label{val4}
\end{align}
From \eqref{val1}, \eqref{val2}, \eqref{val3} and \eqref{val4} we get,
\begin{align}
f(\vec{B}) &= f(3\vec{I}-\vec{e_1}-\vec{e_2})\\
&= 3f(\vec{I})-f(\vec{e_1})-f(\vec{e_2})\\
&= 0\label{Ans}
\end{align}
The \eqref{Ans} is the required answer.
\end{document}
