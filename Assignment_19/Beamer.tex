\documentclass[journal,12pt,twocolumn]{IEEEtran}
%
\usepackage{setspace}
\usepackage{gensymb}
\usepackage{siunitx}
\usepackage{tkz-euclide} 
\usepackage{textcomp}
\usepackage{standalone}
\usetikzlibrary{calc}

%\doublespacing
\singlespacing

%\usepackage{graphicx}
%\usepackage{amssymb}
%\usepackage{relsize}
\usepackage[cmex10]{amsmath}
%\usepackage{amsthm}
%\interdisplaylinepenalty=2500
%\savesymbol{iint}
%\usepackage{txfonts}
%\restoresymbol{TXF}{iint}
%\usepackage{wasysym}
\usepackage{amsthm}
%\usepackage{iithtlc}
\usepackage{mathrsfs}
\usepackage{txfonts}
\usepackage{stfloats}
\usepackage{bm}
\usepackage{cite}
\usepackage{cases}
\usepackage{subfig}
%\usepackage{xtab}
\usepackage{longtable}
\usepackage{multirow}
%\usepackage{algorithm}
%\usepackage{algpseudocode}
\usepackage{enumitem}
\usepackage{mathtools}
\usepackage{steinmetz}
\usepackage{tikz}
\usepackage{circuitikz}
\usepackage{verbatim}
\usepackage{tfrupee}
\usepackage[breaklinks=true]{hyperref}
%\usepackage{stmaryrd}
\usepackage{tkz-euclide} % loads  TikZ and tkz-base
%\usetkzobj{all}
\usetikzlibrary{calc,math}
\usepackage{listings}
    \usepackage{color}                                            %%
    \usepackage{array}                                            %%
    \usepackage{longtable}                                        %%
    \usepackage{calc}                                             %%
    \usepackage{multirow}                                         %%
    \usepackage{hhline}                                           %%
    \usepackage{ifthen}                                           %%
  %optionally (for landscape tables embedded in another document): %%
    \usepackage{lscape}     
\usepackage{multicol}
\usepackage{chngcntr}
\usepackage{amsmath}
\usepackage{cleveref}
%\usepackage{enumerate}

%\usepackage{wasysym}
%\newcounter{MYtempeqncnt}
\DeclareMathOperator*{\Res}{Res}
%\renewcommand{\baselinestretch}{2}
\renewcommand\thesection{\arabic{section}}
\renewcommand\thesubsection{\thesection.\arabic{subsection}}
\renewcommand\thesubsubsection{\thesubsection.\arabic{subsubsection}}

\renewcommand\thesectiondis{\arabic{section}}
\renewcommand\thesubsectiondis{\thesectiondis.\arabic{subsection}}
\renewcommand\thesubsubsectiondis{\thesubsectiondis.\arabic{subsubsection}}

% correct bad hyphenation here
\hyphenation{op-tical net-works semi-conduc-tor}
\def\inputGnumericTable{}                                 %%

\lstset{
%language=C,
frame=single, 
breaklines=true,
columns=fullflexible
}
%\lstset{
%language=tex,
%frame=single, 
%breaklines=true
%}
\usepackage{graphicx}
\usepackage{pgfplots}

\begin{document}
%


\newtheorem{theorem}{Theorem}[section]
\newtheorem{problem}{Problem}
\newtheorem{proposition}{Proposition}[section]
\newtheorem{lemma}{Lemma}[section]
\newtheorem{corollary}[theorem]{Corollary}
\newtheorem{example}{Example}[section]
\newtheorem{definition}[problem]{Definition}
%\newtheorem{thm}{Theorem}[section] 
%\newtheorem{defn}[thm]{Definition}
%\newtheorem{algorithm}{Algorithm}[section]
%\newtheorem{cor}{Corollary}
\newcommand{\BEQA}{\begin{eqnarray}}
\newcommand{\EEQA}{\end{eqnarray}}
\newcommand{\define}{\stackrel{\triangle}{=}}
\bibliographystyle{IEEEtran}
%\bibliographystyle{ieeetr}
\providecommand{\mbf}{\mathbf}
\providecommand{\pr}[1]{\ensuremath{\Pr\left(#1\right)}}
\providecommand{\qfunc}[1]{\ensuremath{Q\left(#1\right)}}
\providecommand{\sbrak}[1]{\ensuremath{{}\left[#1\right]}}
\providecommand{\lsbrak}[1]{\ensuremath{{}\left[#1\right.}}
\providecommand{\rsbrak}[1]{\ensuremath{{}\left.#1\right]}}
\providecommand{\brak}[1]{\ensuremath{\left(#1\right)}}
\providecommand{\lbrak}[1]{\ensuremath{\left(#1\right.}}
\providecommand{\rbrak}[1]{\ensuremath{\left.#1\right)}}
\providecommand{\cbrak}[1]{\ensuremath{\left\{#1\right\}}}
\providecommand{\lcbrak}[1]{\ensuremath{\left\{#1\right.}}
\providecommand{\rcbrak}[1]{\ensuremath{\left.#1\right\}}}
\theoremstyle{remark}
\newtheorem{rem}{Remark}
\newcommand{\sgn}{\mathop{\mathrm{sgn}}}



\providecommand{\abs}[1]{\left\vert#1\right\vert}
\providecommand{\res}[1]{\Res\displaylimits_{#1}} 
\providecommand{\norm}[1]{\left\lVert#1\right\rVert}
%\providecommand{\norm}[1]{\lVert#1\rVert}
\providecommand{\mtx}[1]{\mathbf{#1}}
\providecommand{\mean}[1]{E\left[ #1 \right]}
\providecommand{\fourier}{\overset{\mathcal{F}}{ \rightleftharpoons}}
%\providecommand{\hilbert}{\overset{\mathcal{H}}{ \rightleftharpoons}}
\providecommand{\system}{\overset{\mathcal{H}}{ \longleftrightarrow}}
	%\newcommand{\solution}[2]{\textbf{Solution:}{#1}}
\newcommand{\solution}{\noindent \textbf{Solution: }}
\newcommand{\cosec}{\,\text{cosec}\,}
\providecommand{\dec}[2]{\ensuremath{\overset{#1}{\underset{#2}{\gtrless}}}}
\newcommand{\myvec}[1]{\ensuremath{\begin{pmatrix}#1\end{pmatrix}}}
\newcommand{\mydet}[1]{\ensuremath{\begin{vmatrix}#1\end{vmatrix}}}
%\numberwithin{equation}{section}
\numberwithin{equation}{subsection}
%\numberwithin{problem}{section}
%\numberwithin{definition}{section}
\makeatletter
\@addtoreset{figure}{problem}
\makeatother
\let\StandardTheFigure\thefigure
\let\vec\mathbf
%\renewcommand{\thefigure}{\theproblem.\arabic{figure}}
\renewcommand{\thefigure}{\theproblem}
%\setlist[enumerate,1]{before=\renewcommand\theequation{\theenumi.\arabic{equation}}
%\counterwithin{equation}{enumi}
%\renewcommand{\theequation}{\arabic{subsection}.\arabic{equation}}
\def\putbox#1#2#3{\makebox[0in][l]{\makebox[#1][l]{}\raisebox{\baselineskip}[0in][0in]{\raisebox{#2}[0in][0in]{#3}}}}
     \def\rightbox#1{\makebox[0in][r]{#1}}
     \def\centbox#1{\makebox[0in]{#1}}
     \def\topbox#1{\raisebox{-\baselineskip}[0in][0in]{#1}}
     \def\midbox#1{\raisebox{-0.5\baselineskip}[0in][0in]{#1}}
\vspace{3cm}
\title{Matrix Theory (EE5609) Assignment 19}
\author{Arkadipta De\\MTech Artificial Intelligence\\AI20MTECH14002}

\maketitle
\newpage
%\tableofcontents
\bigskip
\renewcommand{\thefigure}{\theenumi}
\renewcommand{\thetable}{\theenumi}

\begin{abstract}
This document solves a problem on a functional. 
\end{abstract}
All the codes for the figure in this document can be found at
\begin{lstlisting}
https://github.com/Arko98/EE5609/blob/master/Assignment_19
\end{lstlisting}
\section{\textbf{Problem}}
Let $\mathbb{V}$ be the vector space of all $2 \times 2$ matrices over the field of real numbers, and let
\begin{align*}
\vec{B} &= \myvec{2&-2\\-1&1}
\end{align*}
Let $\mathbb{W}$ be the subspace of $\mathbb{V}$ consisting of all $\vec{A}$ such that $\vec{AB} = 0$. Let $f$ be a linear functional on $\mathbb{V}$ which is in the annihilator of $\mathbb{W}$. Suppose that $f(\vec{I}) = 0$ and $f(\vec{C}) = 3$, where $\vec{I}$ is the $2 \times 2$ identity matrix and
\begin{align*}
\vec{C} &= \myvec{0&0\\0&1}
\end{align*}
Find $f(\vec{B})$
\section{\textbf{Solution}}
The general linear functional $f$ on vector space $\mathbb{V}$ is of the form,
\begin{align}
f(\vec{A}) &= \myvec{a&b&c&d}\myvec{x\\y\\z\\w}\label{eq1}
\end{align}
Where,
\begin{align}
\vec{A} &= \myvec{x&y\\z&w} \quad{\text{$\forall$ $\vec{A} \in \mathbb{W}$}}\\
a,b,c,d &\in \mathbb{R}
\end{align}
From $\vec{AB} = 0$ we have,
\begin{align}
\myvec{x&y\\z&w}\myvec{2&-2\\-1&1} = \myvec{0&0\\0&0}\label{eq2}
\end{align}
From \eqref{eq2} we get,
\begin{align}
y &= 2x\label{eqy}\\
w &= 2z\label{eqw}
\end{align}
Hence, using \eqref{eqy} and \eqref{eqw} we conclude that $\mathbb{W}$ consists of all the matrices of the following form,
\begin{align}
\vec{A} &= \myvec{x&2x\\z&2z} \quad{\text{$\forall$ $\vec{A} \in \mathbb{W}$}}\label{eqw}
\end{align}
Hence from \eqref{eqw} we get,
\begin{align}
f\brak{\myvec{x&2x\\z&2z}} &= 0 \quad{\text{$\forall$ $x,z \in \mathbb{R}$}}\\
\implies \myvec{a&b&c&d}\myvec{x\\2x\\z\\2z} &= 0\quad{\text{[From \eqref{eq1}]}}\label{eq3}\\
%\implies (a+2b)x+(c+2d)z &= 0 \quad{\text{$\forall$ $x,z \in \mathbb{R}$}}\label{eq3}
\end{align}
From \eqref{eq3} we get,
\begin{align}
b &= -\frac{1}{2}a\label{eqb}\\
d &= -\frac{1}{2}c\label{eqd}
\end{align}
Hence, from \eqref{eqb}, \eqref{eqd} and \eqref{eq1}, the general form of the functional $f$ on vector space $\mathbb{V}$ becomes,
\begin{align}
f\brak{\vec{A}} &= \myvec{a&-\frac{1}{2}a&c&-\frac{1}{2}c}\myvec{x\\y\\z\\w}\quad{\text{$\forall$ $\vec{A} \in \mathbb{W}$}}\label{eqGen}
\end{align}
Now,
\begin{align}
f(\vec{C}) &= 3\\
\implies \myvec{a&-\frac{1}{2}a&c&-\frac{1}{2}c}\myvec{0\\0\\0\\1} &= 3\\
\implies c &= -6\label{eqVal1}
\end{align}
Again,
\begin{align}
f(\vec{I}) &= 0\\
\implies \myvec{a&-\frac{1}{2}a&c&-\frac{1}{2}c}\myvec{1\\0\\0\\1} &= 0\\
\implies a-\frac{1}{2}c &= 0\\
\implies a &= -3 \quad{\text{[Using \eqref{eqVal1}]}}\label{eqVal2}
\end{align}
Hence, using \eqref{eqVal1} and \eqref{eqVal2} the general form of $f$ in \eqref{eqGen} becomes,
\begin{align}
f\brak{\vec{A}} &= \myvec{-3&\frac{3}{2}&-6&3}\myvec{x\\y\\z\\w} \quad{\text{$\forall$ $\vec{A} \in \mathbb{W}$}}\label{eqGen2}
\end{align}
Now for given $\vec{B}$, from \eqref{eqGen2} we get,
\begin{align}
f\brak{\vec{B}} &= \myvec{-3&\frac{3}{2}&-6&3}\myvec{2\\-2\\-1\\1}\\
\implies f\brak{\vec{B}} &= 0 \label{eqAns}
\end{align}
\eqref{eqAns} is the required answer.
\end{document}
