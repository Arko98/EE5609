\documentclass[journal,12pt,twocolumn]{IEEEtran}

\usepackage{setspace}
\usepackage{gensymb}


\singlespacing

\usepackage[cmex10]{amsmath}
%\usepackage{amsthm}
%\interdisplaylinepenalty=2500
%\savesymbol{iint}
%\usepackage{txfonts}
%\restoresymbol{TXF}{iint}
%\usepackage{wasysym}
\usepackage{amsthm}

\usepackage{mathrsfs}
\usepackage{txfonts}
\usepackage{stfloats}
\usepackage{bm}
\usepackage{cite}
\usepackage{cases}
\usepackage{subfig}

\usepackage{longtable}
\usepackage{multirow}

\usepackage{enumitem}
\usepackage{mathtools}
\usepackage{steinmetz}
\usepackage{tikz}
\usepackage{circuitikz}
\usepackage{verbatim}
\usepackage{tfrupee}
\usepackage[breaklinks=true]{hyperref}

\usepackage{tkz-euclide} %loads TikZ and tkz-base

\usetikzlibrary{calc,math}
\usepackage{listings}
    \usepackage{color}                                          
    \usepackage{array}                                          
    \usepackage{longtable}                                      
    \usepackage{calc}                                           
    \usepackage{multirow}                                       
    \usepackage{hhline}                                         
    \usepackage{ifthen}
    \usepackage{lscape}     
\usepackage{multicol}
\usepackage{chngcntr}

\DeclareMathOperator*{\Res}{Res}

\renewcommand\thesection{\arabic{section}}
\renewcommand\thesubsection{\thesection.\arabic{subsection}}
\renewcommand\thesubsubsection{\thesubsection.\arabic{subsubsection}}

\renewcommand\thesectiondis{\arabic{section}}
\renewcommand\thesubsectiondis{\thesectiondis.\arabic{subsection}}
\renewcommand\thesubsubsectiondis{\thesubsectiondis.\arabic{subsubsection}}

\hyphenation{op-tical net-works semi-conduc-tor}
\def\inputGnumericTable{}                                 %%

\lstset{
%language=C,
frame=single, 
breaklines=true,
columns=fullflexible
}

\begin{document}

\newtheorem{theorem}{Theorem}[section]
\newtheorem{problem}{Problem}
\newtheorem{proposition}{Proposition}[section]
\newtheorem{lemma}{Lemma}[section]
\newtheorem{corollary}[theorem]{Corollary}
\newtheorem{example}{Example}[section]
\newtheorem{definition}[problem]{Definition}

\newcommand{\BEQA}{\begin{eqnarray}}
\newcommand{\EEQA}{\end{eqnarray}}
\newcommand{\define}{\stackrel{\triangle}{=}}
\bibliographystyle{IEEEtran}
\providecommand{\mbf}{\mathbf}
\providecommand{\pr}[1]{\ensuremath{\Pr\left(#1\right)}}
\providecommand{\qfunc}[1]{\ensuremath{Q\left(#1\right)}}
\providecommand{\sbrak}[1]{\ensuremath{{}\left[#1\right]}}
\providecommand{\lsbrak}[1]{\ensuremath{{}\left[#1\right.}}
\providecommand{\rsbrak}[1]{\ensuremath{{}\left.#1\right]}}
\providecommand{\brak}[1]{\ensuremath{\left(#1\right)}}
\providecommand{\lbrak}[1]{\ensuremath{\left(#1\right.}}
\providecommand{\rbrak}[1]{\ensuremath{\left.#1\right)}}
\providecommand{\cbrak}[1]{\ensuremath{\left\{#1\right\}}}
\providecommand{\lcbrak}[1]{\ensuremath{\left\{#1\right.}}
\providecommand{\rcbrak}[1]{\ensuremath{\left.#1\right\}}}
\theoremstyle{remark}
\newtheorem{rem}{Remark}
\newcommand{\sgn}{\mathop{\mathrm{sgn}}}
\providecommand{\abs}[1]{\left\vert#1\right\vert}
\providecommand{\res}[1]{\Res\displaylimits_{#1}} 
\providecommand{\norm}[1]{\left\lVert#1\right\rVert}
%\providecommand{\norm}[1]{\lVert#1\rVert}
\providecommand{\mtx}[1]{\mathbf{#1}}
\providecommand{\mean}[1]{E\left[ #1 \right]}
\providecommand{\fourier}{\overset{\mathcal{F}}{ \rightleftharpoons}}
%\providecommand{\hilbert}{\overset{\mathcal{H}}{ \rightleftharpoons}}
\providecommand{\system}{\overset{\mathcal{H}}{ \longleftrightarrow}}
	%\newcommand{\solution}[2]{\textbf{Solution:}{#1}}
\newcommand{\solution}{\noindent \textbf{Solution: }}
\newcommand{\cosec}{\,\text{cosec}\,}
\providecommand{\dec}[2]{\ensuremath{\overset{#1}{\underset{#2}{\gtrless}}}}
\newcommand{\myvec}[1]{\ensuremath{\begin{pmatrix}#1\end{pmatrix}}}
\newcommand{\mydet}[1]{\ensuremath{\begin{vmatrix}#1\end{vmatrix}}}
\numberwithin{equation}{subsection}
\makeatletter
\@addtoreset{figure}{problem}
\makeatother
\let\StandardTheFigure\thefigure
\let\vec\mathbf
\renewcommand{\thefigure}{\theproblem}
\def\putbox#1#2#3{\makebox[0in][l]{\makebox[#1][l]{}\raisebox{\baselineskip}[0in][0in]{\raisebox{#2}[0in][0in]{#3}}}}
     \def\rightbox#1{\makebox[0in][r]{#1}}
     \def\centbox#1{\makebox[0in]{#1}}
     \def\topbox#1{\raisebox{-\baselineskip}[0in][0in]{#1}}
     \def\midbox#1{\raisebox{-0.5\baselineskip}[0in][0in]{#1}}
\vspace{3cm}
\title{Matrix Theory (EE5609) Challenging Problem}
\author{Arkadipta De\\MTech Artificial Intelligence\\AI20MTECH14002}
\maketitle
\newpage
%\tableofcontents
\bigskip
\renewcommand{\thefigure}{\theenumi}
\renewcommand{\thetable}{\theenumi}
\begin{abstract}
This document proves that $\vec{A^TA}$ has positive eigen values.
\end{abstract}
Download latex codes from 
%
\begin{lstlisting}
https://github.com/Arko98/EE5609/tree/master/Challenge_5
\end{lstlisting}
%
\section{Problem}
Show that the eigen values of $\vec{A^TA}$ are positive. 
\section{Proof}
Let, $\vec{A}$ is an arbitrary $m\times n$ matrix. Now consider the matrix $\vec{A^TA}$,
\begin{align}
\intertext{for any n dimensional vector $\vec{z}$,}
\vec{z^T}(\vec{A^TA})\vec{z} &= \vec{z^TA^T}\vec{Az}\\
\implies\vec{z^T}(\vec{A^TA})\vec{z} &= \vec{({Az})^T}(\vec{Az})\\
\implies\vec{z^T}(\vec{A^TA})\vec{z} &= \norm{\vec{Az}}^2 \geq0\label{eq1}
\intertext{From \eqref{eq1}, if $\vec{z} \not=0$, $\vec{A^TA}$ is positive definite, i.e}
\norm{\vec{Az}}^2 > 0
\intertext{Again, $\vec{A^TA}$ is positive semi-definite, if $\vec{z} = 0$,}
\norm{\vec{Az}}^2 = 0
\end{align}
Hence, $\vec{A^TA}$ is positive semi-definite if the columns of $\vec{A}$ are linearly dependent and $\vec{A^TA}$ is positive definite if columns of $\vec{A}$ are linearly dependent.\\
Again,
\begin{align}
\vec{({A^TA})^T} = \vec{(A^T)(A^T)^T} = \vec{A^TA}\label{eqSym}
\end{align}
Hence, $\vec{A^TA}$ is symmetric. As every eigen value of a Hermitian matrix is real and every symmetric matrix is Hermitian then $\vec{A^TA}$ (being a symmetric and hence Hermitian) has real eigen values.\\
Let $\lambda$ be a (real) eigenvalue of $\vec{B}=\vec{A^TA}$ and let $\vec{x}$ be a corresponding real eigen-vector hence,
\begin{align}
\vec{Bx} = \lambda \vec{x}\label{eq2}
\intertext{Multiplying $\vec{x^T}$ in \eqref{eq2},}
\vec{x^TBx} &= \lambda \vec{x^T}\vec{x}\\
\implies\vec{x^TBx} &= \lambda \norm{\vec{x}}^2\\
\implies\vec{x^T(A^TA)x} &= \lambda \norm{\vec{x}}^2 \qqaud{\text{[$\because\vec{B}=\vec{A^TA}$]}}\label{eqFinal}
\end{align}
When $\vec{A^TA}$ is positive definite (i.e columns of $\vec{A}$ are linearly independent) then, the left hand side of \eqref{eqFinal} is positive as $\vec{A^TA}$ is positive-definite and $\vec{x}$ is a nonzero vector as it is an eigen-vector.\\
Also if $\vec{A}$ has linearly independent columns then $\vec{A^TA}$ will be invertible and hence a non-singular matrix, so $\norm{\vec{x}}$ cannot be zero in \eqref{eqFinal}. Since $\norm{\vec{x}}^2$ is positive, hence all eigen-values must be positive.\\
Hence proved.
\end{document}
