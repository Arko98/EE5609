\documentclass[journal,12pt,twocolumn]{IEEEtran}

\usepackage{setspace}
\usepackage{gensymb}


\singlespacing

\usepackage[cmex10]{amsmath}
%\usepackage{amsthm}
%\interdisplaylinepenalty=2500
%\savesymbol{iint}
%\usepackage{txfonts}
%\restoresymbol{TXF}{iint}
%\usepackage{wasysym}
\usepackage{amsthm}

\usepackage{mathrsfs}
\usepackage{txfonts}
\usepackage{stfloats}
\usepackage{bm}
\usepackage{cite}
\usepackage{cases}
\usepackage{subfig}

\usepackage{longtable}
\usepackage{multirow}

\usepackage{enumitem}
\usepackage{mathtools}
\usepackage{steinmetz}
\usepackage{tikz}
\usepackage{circuitikz}
\usepackage{verbatim}
\usepackage{tfrupee}
\usepackage[breaklinks=true]{hyperref}

\usepackage{tkz-euclide} %loads TikZ and tkz-base

\usetikzlibrary{calc,math}
\usepackage{listings}
    \usepackage{color}                                          
    \usepackage{array}                                          
    \usepackage{longtable}                                      
    \usepackage{calc}                                           
    \usepackage{multirow}                                       
    \usepackage{hhline}                                         
    \usepackage{ifthen}
    \usepackage{lscape}     
\usepackage{multicol}
\usepackage{chngcntr}

\DeclareMathOperator*{\Res}{Res}

\renewcommand\thesection{\arabic{section}}
\renewcommand\thesubsection{\thesection.\arabic{subsection}}
\renewcommand\thesubsubsection{\thesubsection.\arabic{subsubsection}}

\renewcommand\thesectiondis{\arabic{section}}
\renewcommand\thesubsectiondis{\thesectiondis.\arabic{subsection}}
\renewcommand\thesubsubsectiondis{\thesubsectiondis.\arabic{subsubsection}}

\hyphenation{op-tical net-works semi-conduc-tor}
\def\inputGnumericTable{}                                 %%

\lstset{
%language=C,
frame=single, 
breaklines=true,
columns=fullflexible
}

\begin{document}

\newtheorem{theorem}{Theorem}[section]
\newtheorem{problem}{Problem}
\newtheorem{proposition}{Proposition}[section]
\newtheorem{lemma}{Lemma}[section]
\newtheorem{corollary}[theorem]{Corollary}
\newtheorem{example}{Example}[section]
\newtheorem{definition}[problem]{Definition}

\newcommand{\BEQA}{\begin{eqnarray}}
\newcommand{\EEQA}{\end{eqnarray}}
\newcommand{\define}{\stackrel{\triangle}{=}}
\bibliographystyle{IEEEtran}
\providecommand{\mbf}{\mathbf}
\providecommand{\pr}[1]{\ensuremath{\Pr\left(#1\right)}}
\providecommand{\qfunc}[1]{\ensuremath{Q\left(#1\right)}}
\providecommand{\sbrak}[1]{\ensuremath{{}\left[#1\right]}}
\providecommand{\lsbrak}[1]{\ensuremath{{}\left[#1\right.}}
\providecommand{\rsbrak}[1]{\ensuremath{{}\left.#1\right]}}
\providecommand{\brak}[1]{\ensuremath{\left(#1\right)}}
\providecommand{\lbrak}[1]{\ensuremath{\left(#1\right.}}
\providecommand{\rbrak}[1]{\ensuremath{\left.#1\right)}}
\providecommand{\cbrak}[1]{\ensuremath{\left\{#1\right\}}}
\providecommand{\lcbrak}[1]{\ensuremath{\left\{#1\right.}}
\providecommand{\rcbrak}[1]{\ensuremath{\left.#1\right\}}}
\theoremstyle{remark}
\newtheorem{rem}{Remark}
\newcommand{\sgn}{\mathop{\mathrm{sgn}}}
\providecommand{\abs}[1]{\left\vert#1\right\vert}
\providecommand{\res}[1]{\Res\displaylimits_{#1}} 
\providecommand{\norm}[1]{\left\lVert#1\right\rVert}
%\providecommand{\norm}[1]{\lVert#1\rVert}
\providecommand{\mtx}[1]{\mathbf{#1}}
\providecommand{\mean}[1]{E\left[ #1 \right]}
\providecommand{\fourier}{\overset{\mathcal{F}}{ \rightleftharpoons}}
%\providecommand{\hilbert}{\overset{\mathcal{H}}{ \rightleftharpoons}}
\providecommand{\system}{\overset{\mathcal{H}}{ \longleftrightarrow}}
	%\newcommand{\solution}[2]{\textbf{Solution:}{#1}}
\newcommand{\solution}{\noindent \textbf{Solution: }}
\newcommand{\cosec}{\,\text{cosec}\,}
\providecommand{\dec}[2]{\ensuremath{\overset{#1}{\underset{#2}{\gtrless}}}}
\newcommand{\myvec}[1]{\ensuremath{\begin{pmatrix}#1\end{pmatrix}}}
\newcommand{\mydet}[1]{\ensuremath{\begin{vmatrix}#1\end{vmatrix}}}
\numberwithin{equation}{subsection}
\makeatletter
\@addtoreset{figure}{problem}
\makeatother
\let\StandardTheFigure\thefigure
\let\vec\mathbf
\renewcommand{\thefigure}{\theproblem}
\def\putbox#1#2#3{\makebox[0in][l]{\makebox[#1][l]{}\raisebox{\baselineskip}[0in][0in]{\raisebox{#2}[0in][0in]{#3}}}}
     \def\rightbox#1{\makebox[0in][r]{#1}}
     \def\centbox#1{\makebox[0in]{#1}}
     \def\topbox#1{\raisebox{-\baselineskip}[0in][0in]{#1}}
     \def\midbox#1{\raisebox{-0.5\baselineskip}[0in][0in]{#1}}
\vspace{3cm}
\title{Matrix Theory (EE5609) Challenging Problem}
\author{Arkadipta De\\MTech Artificial Intelligence\\AI20MTECH14002}
\maketitle
\newpage
%\tableofcontents
\bigskip
\renewcommand{\thefigure}{\theenumi}
\renewcommand{\thetable}{\theenumi}
\begin{abstract}
This document proves some properties of matrices.
\end{abstract}
Download latex codes from 
%
\begin{lstlisting}
https://github.com/Arko98/EE5609/tree/master/Challenge_7
\end{lstlisting}
%
\section{Problem}
Given two $n \times n$ matrices $\vec{A}$ and $\vec{B}$ prove the following using determinant properties, 
\begin{itemize}
    \item $\det\vec{(AB)} = \det\vec{(A)}\det\vec{(B)}$
    \item $\det\vec{(A^T)} = \det\vec{(A)}$
    \item If $\vec{A}$ is a matrix with orthonormal columns then $|\det\vec{(A)}| = 1$
\end{itemize}
\section{Proof}
\subsection{Proof 1}
\textbf{Case 1:} If $\vec{A}$ is not invertible, then $\vec{AB}$ is not invertible. Hence,
\begin{align}
\det\vec{(AB)} & = 0 \quad{\text{[$\because \vec{AB}$ is not invertible]}}\\
\det\vec{(A)} & = 0 \quad{\text{[$\because \vec{A}$ is not invertible]}}\\
\intertext{Hence,}
\det\vec{(AB)} &= \det\vec{(A)}\det\vec{(B)} = 0\label{proof1a}
\end{align}
\textbf{Case 2:} If $\vec{A}$ is invertible then, there exists a series of elementary row operations $\vec{E_k},\vec{E_{k-1}},\dots\vec{E_1}$ such that,
\begin{align}
    \vec{A} = \vec{E_k}\vec{E_{k-1}}\dots\vec{E_1}\label{eq2}
\end{align}
Now, the determinant of an elementary matrix $\vec{E}$ is given as follows -
\begin{align}
\intertext{If $\vec{E}$ interchanges two rows}
\det(\vec{E}) &= -1\label{e1}
\intertext{If $\vec{E}$ multiplies a row with nonzero constant $c$}
\det(\vec{E}) &= c\label{e2}
\intertext{If $\vec{E}$ multiplies Row $i$ by nonzero constant $c$ and adds to Row $j$}
\det(\vec{E}) &= 1\label{e3}
\end{align}
Now, if $\vec{A}$ interchanges two rows and $\vec{E}$ is the corresponding elementary matrix then,
\begin{align}
\det(\vec{EA}) &= -\det(\vec{A}) \quad{\text{[By property]}}\\
\implies\det(\vec{EA}) &= \det(\vec{E})\det(\vec{A})\quad{\text{[From \eqref{e1}]}}\label{a}
\end{align}
If we multiply the $i$th row of $\vec{A}$ by nonzero constant $c$ and $\vec{E}$ is the corresponding elementary matrix then,
\begin{align}
\det(\vec{EA}) &= c\det(\vec{A}) \quad{\text{[By property]}}\\
\implies\det(\vec{EA}) &= \det(\vec{E})\det(\vec{A})\quad{\text{[From \eqref{e2}]}}\label{b}
\end{align}
Lastly, if we multiply the $i$th row of $\vec{A}$ by nonzero constant $c$ and add it to $j$th row of $\vec{A}$ and $\vec{E}$ is the corresponding elementary matrix then,
\begin{align}
\det(\vec{EA}) &= \det(\vec{A}) \quad{\text{[By property]}}\\
\implies\det(\vec{EA}) &= \det(\vec{E})\det(\vec{A})\quad{\text{[From \eqref{e3}]}}\label{c}
\end{align}
Hence, from \eqref{a}, \eqref{b} and \eqref{c} we get, if elementary matrix $\vec{E}$ represents an elementary row operation and \vec{A} is a $n \times n$ matrix then,
\begin{align}
    \det(\vec{EA}) = \det(\vec{E})\det(\vec{A}) \label{Theorem}
\end{align}
Hence from \eqref{eq2} and \eqref{Theorem} we can write,
\begin{align}
\det\vec{(AB)} &= \det(\vec{E_{k}}\dots\vec{E_1}\vec{B})\\
&= \det(\vec{E_k})\det(\vec{E_{k-1}}\dots\vec{E_1}\vec{B})\\
&= \det(\vec{E_k})\det(\vec{E_{k-1}})\dots\det(\vec{E_1})\det(\vec{B})\\
&= \det(\vec{E_k}\vec{E_{k-1}}\dots\vec{E_1})\det(\vec{B})\\
&=\det\vec{(A)}\det\vec{(B)}\label{proof1b}
\end{align}
Hence from \eqref{proof1a} and \eqref{proof1b} proved.

\subsection{Proof 2}
\textbf{Case 1:} If $rank(\vec{A})<n$ then \vec{A} is not invertible and $\det(\vec{A}) = 0$. Since the row rank and the
column rank are equal, it follows that $\det(\vec{A^T}) = 0$, hence for Case 1, 
\begin{align}
    \det(\vec{A}) = \det(\vec{A^T})\label{proof2a}
\end{align}
\textbf{Case 2:} $\vec{A}$ is invertible. By Gauss elimination $\vec{A}$ can be reduced to the identity matrix, $\vec{I}$ by elementary row operations. Thus $\vec{A}$ is a product of elementary matrices. Hence,
\begin{align}
\vec{A} = \vec{E_1}\vec{E_2}\dots\vec{E_r}\label{eq3}
\end{align}
Now, each elementary matrix is either symmetric or lower-triangular or upper-triangular and by properties of matrices, determinant of a lower-triangular or upper-triangular matrix is product of it's diagonal elements, hence for every elementary matrix $\vec{E_i}$ we have,
\begin{align}
\det(\vec{E_i}) = \det(\vec{E_i^T}) 
\end{align}
Hence from \eqref{eq3} we can write,
\begin{align}
\det(\vec{A}) &= \prod_{i=1}^{r} \det(\vec{E_i})\\
&= \prod_{i=1}^{r} \det(\vec{E_i^T})\\
&= \prod_{i=1}^{r} \det(\vec{E_{r-i}^T})\\
&= \det(\vec{A)}\label{proof2b}
\end{align}
Hence, from \eqref{proof2a} and \eqref{proof2b}, Proved.
\subsection{Proof 3}
If $\vec{A}$ has orthonormal columns then,
\begin{align}
\vec{A^T}\vec{A} = \vec{I}\label{eq4}
\end{align}
Again from properties of determinant we have,
\begin{align}
    \det(\vec{I}) = 1\label{eqI}
\end{align}
Hence, we can write from \eqref{eq4} and \eqref{eqI},
\begin{align}
\det(\vec{I}) &= 1\\
\implies\det(\vec{\vec{A^T}\vec{A}}) &= 1\\
\implies\det(\vec{\vec{A^T})\det(\vec{A}}) &= 1 \quad{\text{[By Proof 1]}}\\
\implies\det(\vec{\vec{A})\det(\vec{A}}) &= 1 \quad{\text{[By Proof 2]}}\\
\implies|\det(\vec{A})| &= 1\label{proof3}
\end{align}
Hence from \eqref{proof3} Proved.
\end{document}
