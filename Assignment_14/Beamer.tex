\documentclass[journal,12pt,twocolumn]{IEEEtran}
%
\usepackage{setspace}
\usepackage{gensymb}
\usepackage{siunitx}
\usepackage{tkz-euclide} 
\usepackage{textcomp}
\usepackage{standalone}
\usetikzlibrary{calc}

%\doublespacing
\singlespacing

%\usepackage{graphicx}
%\usepackage{amssymb}
%\usepackage{relsize}
\usepackage[cmex10]{amsmath}
%\usepackage{amsthm}
%\interdisplaylinepenalty=2500
%\savesymbol{iint}
%\usepackage{txfonts}
%\restoresymbol{TXF}{iint}
%\usepackage{wasysym}
\usepackage{amsthm}
%\usepackage{iithtlc}
\usepackage{mathrsfs}
\usepackage{txfonts}
\usepackage{stfloats}
\usepackage{bm}
\usepackage{cite}
\usepackage{cases}
\usepackage{subfig}
%\usepackage{xtab}
\usepackage{longtable}
\usepackage{multirow}
%\usepackage{algorithm}
%\usepackage{algpseudocode}
\usepackage{enumitem}
\usepackage{mathtools}
\usepackage{steinmetz}
\usepackage{tikz}
\usepackage{circuitikz}
\usepackage{verbatim}
\usepackage{tfrupee}
\usepackage[breaklinks=true]{hyperref}
%\usepackage{stmaryrd}
\usepackage{tkz-euclide} % loads  TikZ and tkz-base
%\usetkzobj{all}
\usetikzlibrary{calc,math}
\usepackage{listings}
    \usepackage{color}                                            %%
    \usepackage{array}                                            %%
    \usepackage{longtable}                                        %%
    \usepackage{calc}                                             %%
    \usepackage{multirow}                                         %%
    \usepackage{hhline}                                           %%
    \usepackage{ifthen}                                           %%
  %optionally (for landscape tables embedded in another document): %%
    \usepackage{lscape}     
\usepackage{multicol}
\usepackage{chngcntr}
\usepackage{amsmath}
\usepackage{cleveref}
%\usepackage{enumerate}

%\usepackage{wasysym}
%\newcounter{MYtempeqncnt}
\DeclareMathOperator*{\Res}{Res}
%\renewcommand{\baselinestretch}{2}
\renewcommand\thesection{\arabic{section}}
\renewcommand\thesubsection{\thesection.\arabic{subsection}}
\renewcommand\thesubsubsection{\thesubsection.\arabic{subsubsection}}

\renewcommand\thesectiondis{\arabic{section}}
\renewcommand\thesubsectiondis{\thesectiondis.\arabic{subsection}}
\renewcommand\thesubsubsectiondis{\thesubsectiondis.\arabic{subsubsection}}

% correct bad hyphenation here
\hyphenation{op-tical net-works semi-conduc-tor}
\def\inputGnumericTable{}                                 %%

\lstset{
%language=C,
frame=single, 
breaklines=true,
columns=fullflexible
}
%\lstset{
%language=tex,
%frame=single, 
%breaklines=true
%}
\usepackage{graphicx}
\usepackage{pgfplots}

\begin{document}
%


\newtheorem{theorem}{Theorem}[section]
\newtheorem{problem}{Problem}
\newtheorem{proposition}{Proposition}[section]
\newtheorem{lemma}{Lemma}[section]
\newtheorem{corollary}[theorem]{Corollary}
\newtheorem{example}{Example}[section]
\newtheorem{definition}[problem]{Definition}
%\newtheorem{thm}{Theorem}[section] 
%\newtheorem{defn}[thm]{Definition}
%\newtheorem{algorithm}{Algorithm}[section]
%\newtheorem{cor}{Corollary}
\newcommand{\BEQA}{\begin{eqnarray}}
\newcommand{\EEQA}{\end{eqnarray}}
\newcommand{\define}{\stackrel{\triangle}{=}}
\bibliographystyle{IEEEtran}
%\bibliographystyle{ieeetr}
\providecommand{\mbf}{\mathbf}
\providecommand{\pr}[1]{\ensuremath{\Pr\left(#1\right)}}
\providecommand{\qfunc}[1]{\ensuremath{Q\left(#1\right)}}
\providecommand{\sbrak}[1]{\ensuremath{{}\left[#1\right]}}
\providecommand{\lsbrak}[1]{\ensuremath{{}\left[#1\right.}}
\providecommand{\rsbrak}[1]{\ensuremath{{}\left.#1\right]}}
\providecommand{\brak}[1]{\ensuremath{\left(#1\right)}}
\providecommand{\lbrak}[1]{\ensuremath{\left(#1\right.}}
\providecommand{\rbrak}[1]{\ensuremath{\left.#1\right)}}
\providecommand{\cbrak}[1]{\ensuremath{\left\{#1\right\}}}
\providecommand{\lcbrak}[1]{\ensuremath{\left\{#1\right.}}
\providecommand{\rcbrak}[1]{\ensuremath{\left.#1\right\}}}
\theoremstyle{remark}
\newtheorem{rem}{Remark}
\newcommand{\sgn}{\mathop{\mathrm{sgn}}}



\providecommand{\abs}[1]{\left\vert#1\right\vert}
\providecommand{\res}[1]{\Res\displaylimits_{#1}} 
\providecommand{\norm}[1]{\left\lVert#1\right\rVert}
%\providecommand{\norm}[1]{\lVert#1\rVert}
\providecommand{\mtx}[1]{\mathbf{#1}}
\providecommand{\mean}[1]{E\left[ #1 \right]}
\providecommand{\fourier}{\overset{\mathcal{F}}{ \rightleftharpoons}}
%\providecommand{\hilbert}{\overset{\mathcal{H}}{ \rightleftharpoons}}
\providecommand{\system}{\overset{\mathcal{H}}{ \longleftrightarrow}}
	%\newcommand{\solution}[2]{\textbf{Solution:}{#1}}
\newcommand{\solution}{\noindent \textbf{Solution: }}
\newcommand{\cosec}{\,\text{cosec}\,}
\providecommand{\dec}[2]{\ensuremath{\overset{#1}{\underset{#2}{\gtrless}}}}
\newcommand{\myvec}[1]{\ensuremath{\begin{pmatrix}#1\end{pmatrix}}}
\newcommand{\mydet}[1]{\ensuremath{\begin{vmatrix}#1\end{vmatrix}}}
%\numberwithin{equation}{section}
\numberwithin{equation}{subsection}
%\numberwithin{problem}{section}
%\numberwithin{definition}{section}
\makeatletter
\@addtoreset{figure}{problem}
\makeatother
\let\StandardTheFigure\thefigure
\let\vec\mathbf
%\renewcommand{\thefigure}{\theproblem.\arabic{figure}}
\renewcommand{\thefigure}{\theproblem}
%\setlist[enumerate,1]{before=\renewcommand\theequation{\theenumi.\arabic{equation}}
%\counterwithin{equation}{enumi}
%\renewcommand{\theequation}{\arabic{subsection}.\arabic{equation}}
\def\putbox#1#2#3{\makebox[0in][l]{\makebox[#1][l]{}\raisebox{\baselineskip}[0in][0in]{\raisebox{#2}[0in][0in]{#3}}}}
     \def\rightbox#1{\makebox[0in][r]{#1}}
     \def\centbox#1{\makebox[0in]{#1}}
     \def\topbox#1{\raisebox{-\baselineskip}[0in][0in]{#1}}
     \def\midbox#1{\raisebox{-0.5\baselineskip}[0in][0in]{#1}}
\vspace{3cm}
\title{Matrix Theory (EE5609) Assignment 14}
\author{Arkadipta De\\MTech Artificial Intelligence\\AI20MTECH14002}

\maketitle
\newpage
%\tableofcontents
\bigskip
\renewcommand{\thefigure}{\theenumi}
\renewcommand{\thetable}{\theenumi}

\begin{abstract}
This document proves the existence of inverse of Hilbert Matrix.  
\end{abstract}
All the codes for the figure in this document can be found at
\begin{lstlisting}
https://github.com/Arko98/EE5609/blob/master/Assignment_14
\end{lstlisting}
\section{\textbf{Problem}}
Prove that the following matrix is invertible and $\vec{A^{-1}}$ has integer entries.\\
\begin{align*}
\vec{A} = \myvec{1&\frac{1}{2}&\dots&\frac{1}{n}\\\frac{1}{2}&\frac{1}{3}&\dots&\frac{1}{n+1}\\\vdots&\vdots&\dots&\vdots\\\frac{1}{n}&\frac{1}{n+1}&\dots&\frac{1}{2n-1}}
\end{align*}
\section{\textbf{Solution}}
\begin{comment}
Let $\vec{H_n}$ be the $n$-th Hilbert matrix given by
\begin{align}
\vec{H_n} &= \left[\frac1{i+j-1}\right]_{i,j}\\
\intertext{Then $\vec{H_{n+1}}$ is given by,}
\vec{H_{n+1}} &= \myvec{\vec{H_n}&\vec{u}\\\vec{u^T}&\frac{1}{2n-1}}
\end{align}
\end{comment}
Let $\vec{A_3}$ be $3 \times 3$ matrix i.e
\begin{align}
\vec{A_3} &= \myvec{1&\frac{1}{2}&\frac{1}{3}\\\frac{1}{2}&\frac{1}{3}&\frac{1}{4}\\\frac{1}{3}&\frac{1}{4}&\frac{1}{5}}
\end{align}
Now we find the inverse of the matrix $\vec{A_3}$ as follows,
\begin{align}
\myvec{1&\frac{1}{2}&\frac{1}{3}&1&0&0\\\frac{1}{2}&\frac{1}{3}&\frac{1}{4}&0&1&0\\\frac{1}{3}&\frac{1}{4}&\frac{1}{5}&0&0&1}\\
\xleftrightarrow[R_3 = R_3 - \frac{1}{3}R_1]{R_2 = R_2 - \frac{1}{2}R_1}\myvec{1&\frac{1}{2}&\frac{1}{3}&1&0&0\\0&\frac{1}{12}&\frac{1}{12}&-\frac{1}{2}&1&0\\0&\frac{1}{12}&\frac{1}{45}&-\frac{1}{3}&0&1}\\
\xleftrightarrow[]{R_3 = R_3 - R_2}\myvec{1&\frac{1}{2}&\frac{1}{3}&1&0&0\\0&\frac{1}{12}&\frac{1}{12}&-\frac{1}{2}&1&0\\0&0&\frac{1}{180}&\frac{1}{6}&-1&1}\\
\xleftrightarrow[R_3 = 180R_3]{R_2 = 12R_2}\myvec{1&\frac{1}{2}&\frac{1}{3}&1&0&0\\0&1&1&-6&12&0\\0&0&1&30&-180&180}\\
\xleftrightarrow[R_1=R_1-R_3]{R_2 = R_2 - R3}\myvec{1&\frac{1}{2}&0&-9&60&-60\\0&1&0&-36&192&-180\\0&0&1&30&-180&180}\\
\xleftrightarrow{R_1 =R_1-\frac{1}{2}R_2}\myvec{1&0&0&9&-36&30\\0&1&0&-36&192&-180\\0&0&1&30&-180&180}
\end{align}
Hence we see that $\vec{A_3}$ is invertible and the inverse contains integer entries and $\vec{A_3^{-1}}$ is given by,
\begin{align}
\vec{A_3^{-1}} = \myvec{9&-36&30\\-36&192&-180\\30&-180&180}\label{A3inv}
\end{align}
Let, $\vec{A_4}$ be $4 \times 4$ matrix as follows,
\begin{align}
\vec{A_4} &= \myvec{1&\frac{1}{2}&\frac{1}{3}&\frac{1}{4}\\\frac{1}{2}&\frac{1}{3}&\frac{1}{4}&\frac{1}{5}\\\frac{1}{3}&\frac{1}{4}&\frac{1}{5}&\frac{1}{6}\\\frac{1}{4}&\frac{1}{5}&\frac{1}{6}&\frac{1}{7}}
\end{align}
Now, expressing $\vec{A_4}$ using $\vec{A_3}$ we get,
\begin{align}
\vec{A_4} &= \myvec{\vec{A_3}&\vec{u}\\\vec{u^T}& d}\label{eqA4}
\intertext{where,}
\vec{u} &= \myvec{\frac{1}{4} \\ \frac{1}{5} \\ \frac{1}{6}} \\
d &= \frac{1}{7}
\end{align}
Now assuming $\vec{A_4}$ has an inverse, then from \eqref{eqA4}, the inverse of $\vec{A_4}$ can be written using block matrix inversion as follows,
\begin{align}
\vec{A_4^{-1}} &= \myvec{\vec{A_3^{-1}}+\vec{A_3^{-1}}\vec{u}x^{-1}\vec{u^T}\vec{A_3^{-1}} & -\vec{A_3^{-1}}\vec{u}x^{-1}\\-x^{-1}\vec{u^T}\vec{A_3^{-1}} & x^{-1}}\label{eqblockinv}
\intertext{where,}
x &= d-{\vec{u^T}\vec{A_3^{-1}}\vec{u}}\label{eqX}
\end{align}
Now, the assumption of $\vec{A_4}$ being invertible will hold if and only if $\vec{A_3}$ is invertible, which has been proved in \eqref{A3inv} and $x$ from \eqref{eqX} is invertible or $x$ is a nonzero scalar. We now prove that $x$ is invertible as follows,
\begin{align}
x &= \frac{1}{7}-\myvec{\frac{1}{4}&\frac{1}{5}&\frac{1}{6}}\myvec{9&-36&30\\-36&192&-180\\30&-180&180}\myvec{\frac{1}{4} \\ \frac{1}{5} \\ \frac{1}{6}}\\
\implies x &= \frac{1}{2800}
\intertext{Hence, $x$ is a scalar, hence $X^{-1}$ exists and is given by,}
x^{-1} &= 2800
\end{align}
Hence, $\vec{A_4}$ is invertible. Now putting the values of $\vec{A_3^{-1}}$, $x^{-1}$ and $\vec{u}$ we get,
\begin{align}
\vec{A_3^{-1}}+\vec{A_3^{-1}}\vec{u}x^{-1}\vec{u^T}\vec{A_3^{-1}} &= \myvec{16&-120&240\\-120&1200&-2700\\240&-2700&6480}\label{A1}\\
-\vec{A_3^{-1}}\vec{u}x^{-1} &= \myvec{-140\\1680\\-4200}\label{A2}\\
x^{-1}\vec{u^T}\vec{A_3^{-1}} &= \myvec{-140&1680&-4200}\label{A3}\\
x^{-1} &= 2800\label{A4}
\end{align}
Putting values from \eqref{A1}, \eqref{A2}, \eqref{A3} and \eqref{A4} into \eqref{eqblockinv} we get,
\begin{align}
\vec{A_4^{-1}} &= \myvec{16&-120&240&-140\\-120&1200&-2700&1680\\240&-2700&6480&-4200\\-140&1680&-4200&2800}\label{A4invfin}
\end{align}
Hence, from \eqref{A4invfin} we prove that, $\vec{A_4}$ is invertible and has integer entries.\\ 
Let $\vec{A_{n-1}}$ be invertible with integer entries. Then we can represent $\vec{A_{n}}$ as follows,
\begin{align}
\vec{A_{n}} &= \myvec{\vec{A_{n-1}}&\vec{u}\\\vec{u^T}&d}\label{eqAn}
\intertext{where,}
\vec{u} &=  \myvec{\frac{1}{4} \\ \frac{1}{5} \\ \vdots \\ \frac{1}{2n-2}} \\
d &= \frac{1}{2n-1}
\end{align}
Now assuming $\vec{A_{n}}$ has an inverse, then from \eqref{eqAn}, the inverse of $\vec{A_n}$ can be written using block matrix inversion as follows,
\begin{align}
\vec{{A_n}^{-1}} &= \myvec{\vec{A_{n-1}^{-1}}+\vec{A_{n-1}^{-1}}\vec{u}x^{-1}\vec{u^T}\vec{A_{n-1}^{-1}} & -\vec{A_{n-1}^{-1}}\vec{u}x^{-1}\\-x^{-1}\vec{u^T}\vec{A_{n-1}^{-1}} & x^{-1}}\label{eqblockinv1}
\intertext{where,}
x &= \vec{d-\vec{u^T}\vec{A_{n-1}^{-1}}\vec{u}}\label{eqX1}
\end{align}
Now, the assumption of $\vec{A_n}$ being invertible will hold if and only if $\vec{A_{n-1}}$ is invertible, which has been assumed and $x$ from \eqref{eqX1} is invertible or $x$ is a nonzero scalar. We now prove that $x$ is invertible as follows,
\begin{align}
x &= \frac{1}{2n-1}-\myvec{\frac{1}{4}&\frac{1}{5}&\dots&\frac{1}{2n-2}}\vec{A_{n-1}^{-1}}\myvec{\frac{1}{4}\\\frac{1}{5}\\\dots\\\frac{1}{2n-2}} \label{eqX2}
\end{align}
In equation \eqref{eqX2} $\vec{u}$ contains no negative or zero entries, again $\vec{A_{n-1}^{-1}}$ has non zero integer entries, hence $\vec{u^T}\vec{A_{n-1}^{-1}}\vec{u}$ is a non zero scalar. Moreover $d$ is not equal to $\vec{u^T}\vec{A_{n-1}^{-1}}\vec{u}$ hence in \eqref{eqX2} $x$ is non-zero scalar and invertible and hence it has an inverse. Hence $\vec{A_n}$ is invertible, proved.
\end{document}
