\documentclass[journal,12pt,twocolumn]{IEEEtran}
%
\usepackage{setspace}
\usepackage{gensymb}
\usepackage{siunitx}
%\doublespacing
\singlespacing

%\usepackage{graphicx}
%\usepackage{amssymb}
%\usepackage{relsize}
\usepackage[cmex10]{amsmath}
%\usepackage{amsthm}
%\interdisplaylinepenalty=2500
%\savesymbol{iint}
%\usepackage{txfonts}
%\restoresymbol{TXF}{iint}
%\usepackage{wasysym}
\usepackage{amsthm}
%\usepackage{iithtlc}
\usepackage{mathrsfs}
\usepackage{txfonts}
\usepackage{stfloats}
\usepackage{bm}
\usepackage{cite}
\usepackage{cases}
\usepackage{subfig}
%\usepackage{xtab}
\usepackage{longtable}
\usepackage{multirow}
%\usepackage{algorithm}
%\usepackage{algpseudocode}
\usepackage{enumitem}
\usepackage{mathtools}
\usepackage{steinmetz}
\usepackage{tikz}
\usepackage{circuitikz}
\usepackage{verbatim}
\usepackage{tfrupee}
\usepackage[breaklinks=true]{hyperref}
%\usepackage{stmaryrd}
\usepackage{tkz-euclide} % loads  TikZ and tkz-base
%\usetkzobj{all}
\usetikzlibrary{calc,math}
\usepackage{listings}
    \usepackage{color}                                            %%
    \usepackage{array}                                            %%
    \usepackage{longtable}                                        %%
    \usepackage{calc}                                             %%
    \usepackage{multirow}                                         %%
    \usepackage{hhline}                                           %%
    \usepackage{ifthen}                                           %%
  %optionally (for landscape tables embedded in another document): %%
    \usepackage{lscape}     
\usepackage{multicol}
\usepackage{chngcntr}
\usepackage{amsmath}

%\usepackage{enumerate}

%\usepackage{wasysym}
%\newcounter{MYtempeqncnt}
\DeclareMathOperator*{\Res}{Res}
%\renewcommand{\baselinestretch}{2}
\renewcommand\thesection{\arabic{section}}
\renewcommand\thesubsection{\thesection.\arabic{subsection}}
\renewcommand\thesubsubsection{\thesubsection.\arabic{subsubsection}}

\renewcommand\thesectiondis{\arabic{section}}
\renewcommand\thesubsectiondis{\thesectiondis.\arabic{subsection}}
\renewcommand\thesubsubsectiondis{\thesubsectiondis.\arabic{subsubsection}}

% correct bad hyphenation here
\hyphenation{op-tical net-works semi-conduc-tor}
\def\inputGnumericTable{}                                 %%

\lstset{
%language=C,
frame=single, 
breaklines=true,
columns=fullflexible
}
%\lstset{
%language=tex,
%frame=single, 
%breaklines=true
%}
\usepackage{graphicx}
\usepackage{pgfplots}

\begin{document}
%


\newtheorem{theorem}{Theorem}[section]
\newtheorem{problem}{Problem}
\newtheorem{proposition}{Proposition}[section]
\newtheorem{lemma}{Lemma}[section]
\newtheorem{corollary}[theorem]{Corollary}
\newtheorem{example}{Example}[section]
\newtheorem{definition}[problem]{Definition}
%\newtheorem{thm}{Theorem}[section] 
%\newtheorem{defn}[thm]{Definition}
%\newtheorem{algorithm}{Algorithm}[section]
%\newtheorem{cor}{Corollary}
\newcommand{\BEQA}{\begin{eqnarray}}
\newcommand{\EEQA}{\end{eqnarray}}
\newcommand{\define}{\stackrel{\triangle}{=}}
\bibliographystyle{IEEEtran}
%\bibliographystyle{ieeetr}
\providecommand{\mbf}{\mathbf}
\providecommand{\pr}[1]{\ensuremath{\Pr\left(#1\right)}}
\providecommand{\qfunc}[1]{\ensuremath{Q\left(#1\right)}}
\providecommand{\sbrak}[1]{\ensuremath{{}\left[#1\right]}}
\providecommand{\lsbrak}[1]{\ensuremath{{}\left[#1\right.}}
\providecommand{\rsbrak}[1]{\ensuremath{{}\left.#1\right]}}
\providecommand{\brak}[1]{\ensuremath{\left(#1\right)}}
\providecommand{\lbrak}[1]{\ensuremath{\left(#1\right.}}
\providecommand{\rbrak}[1]{\ensuremath{\left.#1\right)}}
\providecommand{\cbrak}[1]{\ensuremath{\left\{#1\right\}}}
\providecommand{\lcbrak}[1]{\ensuremath{\left\{#1\right.}}
\providecommand{\rcbrak}[1]{\ensuremath{\left.#1\right\}}}
\theoremstyle{remark}
\newtheorem{rem}{Remark}
\newcommand{\sgn}{\mathop{\mathrm{sgn}}}
\providecommand{\abs}[1]{\left\vert#1\right\vert}
\providecommand{\res}[1]{\Res\displaylimits_{#1}} 
\providecommand{\norm}[1]{\left\lVert#1\right\rVert}
%\providecommand{\norm}[1]{\lVert#1\rVert}
\providecommand{\mtx}[1]{\mathbf{#1}}
\providecommand{\mean}[1]{E\left[ #1 \right]}
\providecommand{\fourier}{\overset{\mathcal{F}}{ \rightleftharpoons}}
%\providecommand{\hilbert}{\overset{\mathcal{H}}{ \rightleftharpoons}}
\providecommand{\system}{\overset{\mathcal{H}}{ \longleftrightarrow}}
	%\newcommand{\solution}[2]{\textbf{Solution:}{#1}}
\newcommand{\solution}{\noindent \textbf{Solution: }}
\newcommand{\cosec}{\,\text{cosec}\,}
\providecommand{\dec}[2]{\ensuremath{\overset{#1}{\underset{#2}{\gtrless}}}}
\newcommand{\myvec}[1]{\ensuremath{\begin{pmatrix}#1\end{pmatrix}}}
\newcommand{\mydet}[1]{\ensuremath{\begin{vmatrix}#1\end{vmatrix}}}
%\numberwithin{equation}{section}
\numberwithin{equation}{subsection}
%\numberwithin{problem}{section}
%\numberwithin{definition}{section}
\makeatletter
\@addtoreset{figure}{problem}
\makeatother
\let\StandardTheFigure\thefigure
\let\vec\mathbf
%\renewcommand{\thefigure}{\theproblem.\arabic{figure}}
\renewcommand{\thefigure}{\theproblem}
%\setlist[enumerate,1]{before=\renewcommand\theequation{\theenumi.\arabic{equation}}
%\counterwithin{equation}{enumi}
%\renewcommand{\theequation}{\arabic{subsection}.\arabic{equation}}
\def\putbox#1#2#3{\makebox[0in][l]{\makebox[#1][l]{}\raisebox{\baselineskip}[0in][0in]{\raisebox{#2}[0in][0in]{#3}}}}
     \def\rightbox#1{\makebox[0in][r]{#1}}
     \def\centbox#1{\makebox[0in]{#1}}
     \def\topbox#1{\raisebox{-\baselineskip}[0in][0in]{#1}}
     \def\midbox#1{\raisebox{-0.5\baselineskip}[0in][0in]{#1}}
\vspace{3cm}
\title{Matrix Theory (EE5609) Assignment 4}
\author{Arkadipta De\\MTech Artificial Intelligence\\AI20MTECH14002}

\maketitle
\newpage
%\tableofcontents
\bigskip
\renewcommand{\thefigure}{\theenumi}
\renewcommand{\thetable}{\theenumi}

\begin{abstract}
This document solves an equation on matrix. Additionally it finds characteristic equation of a square matrix.
\end{abstract}
\begin{comment}
The code for the solution of this problem can be found at
%
\begin{lstlisting}
https://github.com/Arko98/EE5609/blob/master/Assignment_4/Codes/Solution.py
\end{lstlisting}
%
\end{comment}
\section{Problem}
If $\vec{A}$ = \myvec{1&0&2\\0&2&1\\2&0&3}, prove that $\vec{A^3}-6\vec{A^2}+7\vec{A}+2\vec{I}$=0
\section{Solution}
The solution code for this problem can be found at: \url{https://github.com/Arko98/EE5609/blob/master/Assignment_4/Codes/Solution.py}
\begin{comment}
The equation in the problem can be modified as follows
\begin{align}
\vec{A^3}-6\vec{A^2}+7\vec{A}+2\vec{I}&=0\\
\implies\vec{A^2}(\vec{A^}-6\vec{I})+7\vec{A}+2\vec{I}&=0\label{eq1}
\end{align}
So we need to prove equation \ref{eq1}. Now, at first we calculate the value of $\vec{A^2}$ as follows
\begin{align}
\vec{A^2}&=\vec{A}\cdot\vec{A}\\
\implies\vec{A^2}&=\vec{A^T}\vec{A}\\
\implies\vec{A^2}&=\myvec{1&0&2\\0&2&0\\2&1&3}\myvec{1&0&2\\0&2&1\\2&0&3}\\
\implies\vec{A^2}&=\myvec{5&0&8\\2&4&5\\8&0&13}\label{eq2}
\end{align}
Next we calculate, $\vec{A}-6\vec{I}$ where I is identity matrix of order 3, as follows
\begin{align}
\vec{A}-6\vec{I}&=\myvec{1&0&2\\0&2&1\\2&0&3} - 6\myvec{1&0&0\\0&1&0\\0&0&1}\\
\implies\vec{A}-6\vec{I}&=\myvec{1&0&2\\0&2&1\\2&0&3} - \myvec{6&0&0\\0&6&0\\0&0&6}\\
\implies\vec{A}-6\vec{I}&=\myvec{-5&0&2\\0&-4&1\\2&0&-3}\label{eq3}
\end{align}
Now we compute $\vec{A^2}(\vec{A}-6\vec{I})$ by putting values of $\vec{A^2}$ from equation \ref{eq2} and value of $\vec{A}-6\vec{I}$ from \ref{eq3} as follows
\begin{align}
\vec{A^2}\cdot(\vec{A}-6\vec{I})&=\vec{(A^2)^T}(\vec{A}-6\vec{I})\\
\implies\vec{A^2}\cdot(\vec{A}-6\vec{I})&=\myvec{5&2&8\\0&4&0\\8&5&13}\myvec{-5&0&2\\0&-4&1\\2&0&-3}\\
\implies\vec{A^2}\cdot(\vec{A}-6\vec{I})&=\myvec{-9&0&-14\\0&-16&-7\\-14&0&-23}\label{eq4}
\end{align}
Next we compute $7\vec{A}+2\vec{I}$ as follows
\begin{align}
7\vec{A}+2\vec{I} &= 7\myvec{1&0&2\\0&2&1\\2&0&3}+2\myvec{1&0&0\\0&1&0\\0&0&1}\\
\implies7\vec{A}+2\vec{I}&=\myvec{9&0&14\\0&16&7\\14&0&23}\label{eq5}
\end{align}
Now putting the value of $\vec{A^2}(\vec{A}-6\vec{I})$ from equation \ref{eq4} and value of $7\vec{A}+2\vec{I}$ from equation \ref{eq5} into the left hand side of equation \ref{eq1} we get
\begin{align}\textwidth
\vec{A^2}(\vec{A^}-6\vec{I})+7\vec{A}+2\vec{I}&=\myvec{0&0&0\\0&0&0\\0&0&0}\\
\implies\vec{A^2}(\vec{A^}-6\vec{I})+7\vec{A}+2\vec{I}&=0\label{eqF}
\end{align}
Thus from equation \ref{eqF} we arrive at the right hand side of equation \ref{eq1}, Hence proved.
\end{comment}
\section{Problem 2}
Find the characteristic equation of the matrix $\vec{A}$ = \myvec{1&0&2\\0&2&1\\2&0&3}
\section{Solution}
For a general order $k$ square matrix $\vec{A}$, the characteristic equation in variable $\lambda$ is defined by
\begin{align}
\det({\vec{A}-\lambda \vec{I}}) &= 0\label{eqChar1}\\
\implies\det{\vec{A}}-\lambda \det{\vec{I}} &= 0\\
\implies\mydet{1&0&2\\0&2&1\\2&0&3}-\mydet{\lambda &0&0\\0&\lambda &0\\0&0&\lambda } &= 0\\
\implies\mydet{1-\lambda &0&2\\0&2-\lambda &1\\2&0& 3-\lambda} &= 0\label{eqFinal}
\end{align}
Hence, (\ref{eqFinal}) is the required characteristic equation of matrix $\vec{A}$. Further expanding the determinant from (\ref{eqFinal}) we get the following polynomial equation of $\lambda$ 
\begin{align}
(1-\lambda )(2-\lambda)(3-\lambda)-4(2-\lambda) &= 0\\
\implies\lambda^3-6\lambda^2+7\lambda+2 &=0\label{eqpol}
\end{align}
Hence, (\ref{eqpol}) is the required characteristic equation of matrix $\vec{A}$ for $\lambda$.
\end{document}
