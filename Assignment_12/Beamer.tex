\documentclass[journal,12pt,twocolumn]{IEEEtran}
%
\usepackage{setspace}
\usepackage{gensymb}
\usepackage{siunitx}
\usepackage{tkz-euclide} 
\usepackage{textcomp}
\usepackage{standalone}
\usetikzlibrary{calc}

%\doublespacing
\singlespacing

%\usepackage{graphicx}
%\usepackage{amssymb}
%\usepackage{relsize}
\usepackage[cmex10]{amsmath}
%\usepackage{amsthm}
%\interdisplaylinepenalty=2500
%\savesymbol{iint}
%\usepackage{txfonts}
%\restoresymbol{TXF}{iint}
%\usepackage{wasysym}
\usepackage{amsthm}
%\usepackage{iithtlc}
\usepackage{mathrsfs}
\usepackage{txfonts}
\usepackage{stfloats}
\usepackage{bm}
\usepackage{cite}
\usepackage{cases}
\usepackage{subfig}
%\usepackage{xtab}
\usepackage{longtable}
\usepackage{multirow}
%\usepackage{algorithm}
%\usepackage{algpseudocode}
\usepackage{enumitem}
\usepackage{mathtools}
\usepackage{steinmetz}
\usepackage{tikz}
\usepackage{circuitikz}
\usepackage{verbatim}
\usepackage{tfrupee}
\usepackage[breaklinks=true]{hyperref}
%\usepackage{stmaryrd}
\usepackage{tkz-euclide} % loads  TikZ and tkz-base
%\usetkzobj{all}
\usetikzlibrary{calc,math}
\usepackage{listings}
    \usepackage{color}                                            %%
    \usepackage{array}                                            %%
    \usepackage{longtable}                                        %%
    \usepackage{calc}                                             %%
    \usepackage{multirow}                                         %%
    \usepackage{hhline}                                           %%
    \usepackage{ifthen}                                           %%
  %optionally (for landscape tables embedded in another document): %%
    \usepackage{lscape}     
\usepackage{multicol}
\usepackage{chngcntr}
\usepackage{amsmath}
\usepackage{cleveref}
%\usepackage{enumerate}

%\usepackage{wasysym}
%\newcounter{MYtempeqncnt}
\DeclareMathOperator*{\Res}{Res}
%\renewcommand{\baselinestretch}{2}
\renewcommand\thesection{\arabic{section}}
\renewcommand\thesubsection{\thesection.\arabic{subsection}}
\renewcommand\thesubsubsection{\thesubsection.\arabic{subsubsection}}

\renewcommand\thesectiondis{\arabic{section}}
\renewcommand\thesubsectiondis{\thesectiondis.\arabic{subsection}}
\renewcommand\thesubsubsectiondis{\thesubsectiondis.\arabic{subsubsection}}

% correct bad hyphenation here
\hyphenation{op-tical net-works semi-conduc-tor}
\def\inputGnumericTable{}                                 %%

\lstset{
%language=C,
frame=single, 
breaklines=true,
columns=fullflexible
}
%\lstset{
%language=tex,
%frame=single, 
%breaklines=true
%}
\usepackage{graphicx}
\usepackage{pgfplots}

\begin{document}
%


\newtheorem{theorem}{Theorem}[section]
\newtheorem{problem}{Problem}
\newtheorem{proposition}{Proposition}[section]
\newtheorem{lemma}{Lemma}[section]
\newtheorem{corollary}[theorem]{Corollary}
\newtheorem{example}{Example}[section]
\newtheorem{definition}[problem]{Definition}
%\newtheorem{thm}{Theorem}[section] 
%\newtheorem{defn}[thm]{Definition}
%\newtheorem{algorithm}{Algorithm}[section]
%\newtheorem{cor}{Corollary}
\newcommand{\BEQA}{\begin{eqnarray}}
\newcommand{\EEQA}{\end{eqnarray}}
\newcommand{\define}{\stackrel{\triangle}{=}}
\bibliographystyle{IEEEtran}
%\bibliographystyle{ieeetr}
\providecommand{\mbf}{\mathbf}
\providecommand{\pr}[1]{\ensuremath{\Pr\left(#1\right)}}
\providecommand{\qfunc}[1]{\ensuremath{Q\left(#1\right)}}
\providecommand{\sbrak}[1]{\ensuremath{{}\left[#1\right]}}
\providecommand{\lsbrak}[1]{\ensuremath{{}\left[#1\right.}}
\providecommand{\rsbrak}[1]{\ensuremath{{}\left.#1\right]}}
\providecommand{\brak}[1]{\ensuremath{\left(#1\right)}}
\providecommand{\lbrak}[1]{\ensuremath{\left(#1\right.}}
\providecommand{\rbrak}[1]{\ensuremath{\left.#1\right)}}
\providecommand{\cbrak}[1]{\ensuremath{\left\{#1\right\}}}
\providecommand{\lcbrak}[1]{\ensuremath{\left\{#1\right.}}
\providecommand{\rcbrak}[1]{\ensuremath{\left.#1\right\}}}
\theoremstyle{remark}
\newtheorem{rem}{Remark}
\newcommand{\sgn}{\mathop{\mathrm{sgn}}}



\providecommand{\abs}[1]{\left\vert#1\right\vert}
\providecommand{\res}[1]{\Res\displaylimits_{#1}} 
\providecommand{\norm}[1]{\left\lVert#1\right\rVert}
%\providecommand{\norm}[1]{\lVert#1\rVert}
\providecommand{\mtx}[1]{\mathbf{#1}}
\providecommand{\mean}[1]{E\left[ #1 \right]}
\providecommand{\fourier}{\overset{\mathcal{F}}{ \rightleftharpoons}}
%\providecommand{\hilbert}{\overset{\mathcal{H}}{ \rightleftharpoons}}
\providecommand{\system}{\overset{\mathcal{H}}{ \longleftrightarrow}}
	%\newcommand{\solution}[2]{\textbf{Solution:}{#1}}
\newcommand{\solution}{\noindent \textbf{Solution: }}
\newcommand{\cosec}{\,\text{cosec}\,}
\providecommand{\dec}[2]{\ensuremath{\overset{#1}{\underset{#2}{\gtrless}}}}
\newcommand{\myvec}[1]{\ensuremath{\begin{pmatrix}#1\end{pmatrix}}}
\newcommand{\mydet}[1]{\ensuremath{\begin{vmatrix}#1\end{vmatrix}}}
%\numberwithin{equation}{section}
\numberwithin{equation}{subsection}
%\numberwithin{problem}{section}
%\numberwithin{definition}{section}
\makeatletter
\@addtoreset{figure}{problem}
\makeatother
\let\StandardTheFigure\thefigure
\let\vec\mathbf
%\renewcommand{\thefigure}{\theproblem.\arabic{figure}}
\renewcommand{\thefigure}{\theproblem}
%\setlist[enumerate,1]{before=\renewcommand\theequation{\theenumi.\arabic{equation}}
%\counterwithin{equation}{enumi}
%\renewcommand{\theequation}{\arabic{subsection}.\arabic{equation}}
\def\putbox#1#2#3{\makebox[0in][l]{\makebox[#1][l]{}\raisebox{\baselineskip}[0in][0in]{\raisebox{#2}[0in][0in]{#3}}}}
     \def\rightbox#1{\makebox[0in][r]{#1}}
     \def\centbox#1{\makebox[0in]{#1}}
     \def\topbox#1{\raisebox{-\baselineskip}[0in][0in]{#1}}
     \def\midbox#1{\raisebox{-0.5\baselineskip}[0in][0in]{#1}}
\vspace{3cm}
\title{Matrix Theory (EE5609) Assignment 12}
\author{Arkadipta De\\MTech Artificial Intelligence\\AI20MTECH14002}

\maketitle
\newpage
%\tableofcontents
\bigskip
\renewcommand{\thefigure}{\theenumi}
\renewcommand{\thetable}{\theenumi}

\begin{abstract}
This document proves that, each field of the characteristic zero contains a copy of the rational number field.
\end{abstract}

All the codes for the figure in this document can be found at
\begin{lstlisting}
https://github.com/Arko98/EE5609/blob/master/Assignment_12
\end{lstlisting}

\section{\textbf{Problem}}
Consider the system of equations $\vec{AX}$ = 0 where
\begin{align*}
    \vec{A} = \myvec{a&b\\c&d}
\end{align*}
is a $2\times2$ matrix over the field $F$. Prove the following - 
\begin{itemize}
    \item If every entry of $\vec{A}$ is 0, then every pair $x_1$ and $x_2$ is a solution of $\vec{AX}$ = 0.
    \item If $ad - bc \not=$ 0, then the system $\vec{AX}$ = 0 has only the trivial solution $x_1 = x_2 = 0$
    \item If $ad - bc = 0$ and some entry of $\vec{A}$ is different from 0, then there is a solution $x_1^0$ and $x_2^0$ such that $x_1$ and  $x_2$ is a solution if and only if there is a scalar $y$ such that $x_1 = yx_1^0$ and $x_2 = yx_2^0$
\end{itemize}
\section{\textbf{Solution}}
\subsection{Solution 1}
If every entry of $\vec{A}$ is 0 then the equation $\vec{AX}$ = 0 becomes,
\begin{align}
\myvec{0&0\\0&0}\myvec{x_1\\x_2} &=0\\
\implies0.x_1+0.x_2 &= 0 \qquad{\text{$\forall x_1,x_2 \in F$}}
\end{align}
Hence proved, every pair $x_1$ and $x_2$ is a solution for the equation $\vec{AX}$ = 0.
\subsection{Solution 2}
\textbf{Case 1: }Let $a=0$. Since $ad-bc\not=0$. As $bc\not=0$ therefore $b\not=0$ and $c\not=0$. Hence, we can perform row reduction on the augmented matrix of equation $\vec{AX}$=0 as follows,
\begin{align}
\myvec{0&b&0\\c&d&0}&\xleftrightarrow{R_1\leftrightarrow R_2}\myvec{c&d&0\\0&b&0}\\
&\xleftrightarrow[R_2 =\frac{1}{b}R_2]{R_1 =\frac{1}{c}R_1}\myvec{1&\frac{d}{c}&0\\0&1&0}\\
&\xleftrightarrow{R_1=R_1 - \frac{d}{c}R_2}\myvec{1&0&0\\0&1&0}\label{eqI1}
\end{align}
\begin{comment}

\textbf{Case 2: }Let $a\not=0$. Hence, we can perform row reduction on the augmented matrix of equation $\vec{AX}$=0 as follows,
\begin{align}
\myvec{a&b&0\\c&d&0}&\xleftrightarrow{R_1 = \frac{1}{a}R_1}\myvec{1&\frac{b}{a}&0\\c&d&0}\\
&\xleftrightarrow{R_2 = R_2-cR_1}\myvec{1&\frac{b}{a}&0\\0&\frac{ad-bc}{a}&0}\\
&\xleftrightarrow{R_2=\frac{a}{ad-bc}R_2}\myvec{1&\frac{b}{a}&0\\0&1&0}\\
&\xleftrightarrow{R_1 = R_1-\frac{b}{a}R_2}\myvec{1&0&0\\0&1&0}\label{eqI2}
\end{align}
\textbf{Case 3: }Let $a,b,c,d \not= 0$. Hence, we can perform row reduction on the augmented matrix of equation $\vec{AX}$=0 as follows,
\begin{align}
\myvec{a&b&0\\c&d&0}&\xleftrightarrow{R_1 = \frac{1}{a}R_1}\myvec{1&\frac{b}{a}&0\\c&d&0}\\
&\xleftrightarrow{R_2 = R_2-cR_1}\myvec{1&\frac{b}{a}&0\\0&\frac{ad-bc}{a}&0}\\
&\xleftrightarrow{R_2=\frac{a}{ad-bc}R_2}\myvec{1&\frac{b}{a}&0\\0&1&0}\\
&\xleftrightarrow{R_1 = R_1-\frac{b}{a}R_2}\myvec{1&0&0\\0&1&0}\label{eqI3}
\end{align}

\end{comment}
\textbf{Case 2: }Let $a,b,c,d\not=0$. Considering the following case,
\begin{align}
\vec{A}\vec{X} = \vec{u}\\
\implies\myvec{a&b\\c&d}\myvec{x_1\\x_2} &=\myvec{u_1\\u_2}\label{eqComp}\\
\intertext{Row Reducing the augmented matrix of \eqref{eqComp} we get,}
\myvec{a&b&u_1\\c&d&u_2}&\xleftrightarrow{R_1 = \frac{1}{a}R_1}\myvec{1&\frac{b}{a}&\frac{u_1}{a}\\c&d&u_2}\\
&\xleftrightarrow{R_2 = R_2-cR_1}\myvec{1&\frac{b}{a}&\frac{u_1}{a}\\0&\frac{ad-bc}{a}&\frac{au_2-cu_1}{a}}\\
&\xleftrightarrow{R_2=\frac{a}{ad-bc}R_2}\myvec{1&\frac{b}{a}&\frac{u_1}{a}\\0&1&\frac{au_2-cu_1}{ad-bc}}\\
&\xleftrightarrow{R_1=R_1-\frac{b}{a}R_2}\myvec{1&0&\frac{du_1-bu_2}{ad-bc}\\0&1&\frac{au_2-cu_1}{ad-bc}}\label{eqsolve}\\
\intertext{From \eqref{eqsolve} we get,}
x_1 &= \frac{du_1-bu_2}{ad-bc}\label{eqU1}\\
x_2 &= \frac{au_2-cu_1}{ad-bc}\label{eqU2}\\
\intertext{Since $u_1 = 0$ and $u_2 = 0$ then from \eqref{eqU1} and \eqref{eqU2},}
x_1 & = 0\\
x_2 & = 0\\
\intertext{Hence we get,}
\vec{x} = \myvec{x_1\\x_2} &= \myvec{0\\0}\label{eqI4}
\end{align}
In \eqref{eqI1} and \eqref{eqI4}, we can see that $\vec{A}\vec{X}=0$ has only one trivial solution i.e $x_1=x_2=0$ in all cases. Hence proved, the equation $\vec{AX}$=0 has only one trivial solution $x_1=x_2=0$
\subsection{Solution 3}
\textbf{Case 1: }Let, $a\not=0$ for $\vec{A}$. Given $ad-bc=0$, we can perform row reduction on augmented matrix of equation $\vec{AX}=0$ as follows,
\begin{align}
\myvec{a&b&0\\c&d&0}&\xleftrightarrow{R_1 = \frac{1}{a}R_1}\myvec{1&\frac{b}{a}&0\\c&d&0}\\
&\xleftrightarrow{R_2 = R_2-cR_1}\myvec{1&\frac{b}{a}&0\\0&0&0}\quad{\text{[$\because ad-bc=0$]}}\label{eq3}
\end{align}
Hence from \eqref{eq3}, $\vec{AX}$ = 0 if and only if 
\begin{align}
x_1 &= -\frac{b}{a}x_2 \qquad{\text{[$a\not=0$]}}
\intertext{Letting $x_1^0=-\frac{b}{a}$ and $x_2^0 = 1$ we get for $y=1$,}
x_1 &= yx_1^0\\
x_2 &= yx_2^0
\end{align}
which is a solution of the equation $\vec{AX}=0$. \\
\textbf{Case 2: }Let, $b\not=0$ for $\vec{A}$. Given $ad-bc=0$, we can perform row reduction on augmented matrix of equation $\vec{AX}=0$ as follows,
\begin{align}
\myvec{a&b&0\\c&d&0}&\xleftrightarrow{C_1 \leftrightarrow C_2}\myvec{b&a&0\\d&c&0}\\
&\xleftrightarrow{R_1 = \frac{1}{b}R_1}\myvec{1&\frac{a}{b}&0\\d&c&0}\\
&\xleftrightarrow{R_2 = R_2-dR_1}\myvec{1&\frac{a}{b}&0\\0&0&0}\quad{\text{[$\because ad-bc=0$]}}\label{eq3}
\end{align}
Hence from \eqref{eq3}, $\vec{AX}$ = 0 if and only if 
\begin{align}
x_2 &= -\frac{a}{b}x_1 \qquad{\text{[$b\not=0$]}}
\intertext{Letting $x_2^0=-\frac{a}{b}$ and $x_1^0 = 1$ we get for $y=1$,}
x_1 &= yx_1^0\\
x_2 &= yx_2^0
\end{align}
which is a solution of the equation $\vec{AX}=0$. \\
\textbf{Case 3: }Let, $c\not=0$ for $\vec{A}$. Given $ad-bc=0$, we can perform row reduction on augmented matrix of equation $\vec{AX}=0$ as follows,
\begin{align}
\myvec{a&b&0\\c&d&0}&\xleftrightarrow{R_1 \leftrightarrow R_2}\myvec{c&d&0\\a&b&0}\\
&\xleftrightarrow{R_1 = \frac{1}{c}R_1}\myvec{1&\frac{d}{c}&0\\a&b&0}\\
&\xleftrightarrow{R_2 = R_2-aR_1}\myvec{1&\frac{d}{c}&0\\0&0&0}\quad{\text{[$\because ad-bc=0$]}}\label{eq3}
\end{align}
Hence from \eqref{eq3}, $\vec{AX}$ = 0 if and only if 
\begin{align}
x_1 &= -\frac{d}{c}x_2 \qquad{\text{[$a\not=0$]}}
\intertext{Letting $x_1^0=-\frac{d}{c}$ and $x_2^0 = 1$ we get for $y=1$,}
x_1 &= yx_1^0\\
x_2 &= yx_2^0
\end{align}
which is a solution of the equation $\vec{AX}=0$. \\
\textbf{Case 4: }Let, $d\not=0$ for $\vec{A}$. Given $ad-bc=0$, we can perform row reduction on augmented matrix of equation $\vec{AX}=0$ as follows,
\begin{align}
\myvec{a&b&0\\c&d&0}&\xleftrightarrow{R_1 \leftrightarrow R_2}\myvec{c&d&0\\a&b&0}\\
&\xleftrightarrow{C_1 \leftrightarrow C_2}\myvec{d&c&0\\b&a&0}\\
&\xleftrightarrow{R_1 = \frac{1}{d}R_1}\myvec{1&\frac{c}{d}&0\\b&a&0}\\
&\xleftrightarrow{R_2 = R_2-bR_1}\myvec{1&\frac{c}{d}&0\\0&0&0}\quad{\text{[$\because ad-bc=0$]}}\label{eq3}
\end{align}
Hence from \eqref{eq3}, $\vec{AX}$ = 0 if and only if 
\begin{align}
x_2 &= -\frac{d}{c}x_1 \qquad{\text{[$a\not=0$]}}
\intertext{Letting $x_2^0=-\frac{d}{c}$ and $x_1^0 = 1$ we get for $y=1$,}
x_1 &= yx_1^0\\
x_2 &= yx_2^0
\end{align}
which is a solution of the equation $\vec{AX}=0$. \\
Hence Proved.
\end{document}
