\documentclass[journal,12pt,twocolumn]{IEEEtran}

\usepackage{setspace}
\usepackage{gensymb}

\singlespacing


\usepackage[cmex10]{amsmath}

\usepackage{amsthm}
\usepackage{hyperref}
\usepackage{mathrsfs}
\usepackage{txfonts}
\usepackage{stfloats}
\usepackage{bm}
\usepackage{cite}
\usepackage{cases}
\usepackage{subfig}

\usepackage{longtable}
\usepackage{multirow}

\usepackage{enumitem}
\usepackage{mathtools}
\usepackage{steinmetz}
\usepackage{tikz}
\usepackage{circuitikz}
\usepackage{verbatim}
\usepackage{tfrupee}
\usepackage[breaklinks=true]{hyperref}

\usepackage{tkz-euclide}

\usetikzlibrary{calc,math}
\usepackage{listings}
    \usepackage{color}                                            %%
    \usepackage{array}                                            %%
    \usepackage{longtable}                                        %%
    \usepackage{calc}                                             %%
    \usepackage{multirow}                                         %%
    \usepackage{hhline}                                           %%
    \usepackage{ifthen}                                           %%
    \usepackage{lscape}     
\usepackage{multicol}
\usepackage{chngcntr}

\DeclareMathOperator*{\Res}{Res}

\renewcommand\thesection{\arabic{section}}
\renewcommand\thesubsection{\thesection.\arabic{subsection}}
\renewcommand\thesubsubsection{\thesubsection.\arabic{subsubsection}}

\renewcommand\thesectiondis{\arabic{section}}
\renewcommand\thesubsectiondis{\thesectiondis.\arabic{subsection}}
\renewcommand\thesubsubsectiondis{\thesubsectiondis.\arabic{subsubsection}}


\hyphenation{op-tical net-works semi-conduc-tor}
\def\inputGnumericTable{}                                 %%

\lstset{
%language=C,
frame=single, 
breaklines=true,
columns=fullflexible
}
\begin{document}


\newtheorem{theorem}{Theorem}[section]
\newtheorem{problem}{Problem}
\newtheorem{proposition}{Proposition}[section]
\newtheorem{lemma}{Lemma}[section]
\newtheorem{corollary}[theorem]{Corollary}
\newtheorem{example}{Example}[section]
\newtheorem{definition}[problem]{Definition}

\newcommand{\BEQA}{\begin{eqnarray}}
\newcommand{\EEQA}{\end{eqnarray}}
\newcommand{\define}{\stackrel{\triangle}{=}}
\bibliographystyle{IEEEtran}
\providecommand{\mbf}{\mathbf}
\providecommand{\pr}[1]{\ensuremath{\Pr\left(#1\right)}}
\providecommand{\qfunc}[1]{\ensuremath{Q\left(#1\right)}}
\providecommand{\sbrak}[1]{\ensuremath{{}\left[#1\right]}}
\providecommand{\lsbrak}[1]{\ensuremath{{}\left[#1\right.}}
\providecommand{\rsbrak}[1]{\ensuremath{{}\left.#1\right]}}
\providecommand{\brak}[1]{\ensuremath{\left(#1\right)}}
\providecommand{\lbrak}[1]{\ensuremath{\left(#1\right.}}
\providecommand{\rbrak}[1]{\ensuremath{\left.#1\right)}}
\providecommand{\cbrak}[1]{\ensuremath{\left\{#1\right\}}}
\providecommand{\lcbrak}[1]{\ensuremath{\left\{#1\right.}}
\providecommand{\rcbrak}[1]{\ensuremath{\left.#1\right\}}}
\theoremstyle{remark}
\newtheorem{rem}{Remark}
\newcommand{\sgn}{\mathop{\mathrm{sgn}}}
\providecommand{\abs}[1]{\left\vert#1\right\vert}
\providecommand{\res}[1]{\Res\displaylimits_{#1}} 
\providecommand{\norm}[1]{\left\lVert#1\right\rVert}
%\providecommand{\norm}[1]{\lVert#1\rVert}
\providecommand{\mtx}[1]{\mathbf{#1}}
\providecommand{\mean}[1]{E\left[ #1 \right]}
\providecommand{\fourier}{\overset{\mathcal{F}}{ \rightleftharpoons}}
%\providecommand{\hilbert}{\overset{\mathcal{H}}{ \rightleftharpoons}}
\providecommand{\system}{\overset{\mathcal{H}}{ \longleftrightarrow}}
	%\newcommand{\solution}[2]{\textbf{Solution:}{#1}}
\newcommand{\solution}{\noindent \textbf{Solution: }}
\newcommand{\cosec}{\,\text{cosec}\,}
\providecommand{\dec}[2]{\ensuremath{\overset{#1}{\underset{#2}{\gtrless}}}}
\newcommand{\myvec}[1]{\ensuremath{\begin{pmatrix}#1\end{pmatrix}}}
\newcommand{\mydet}[1]{\ensuremath{\begin{vmatrix}#1\end{vmatrix}}}
\numberwithin{equation}{subsection}
\makeatletter
\@addtoreset{figure}{problem}
\makeatother
\let\StandardTheFigure\thefigure
\let\vec\mathbf
\renewcommand{\thefigure}{\theproblem}
\def\putbox#1#2#3{\makebox[0in][l]{\makebox[#1][l]{}\raisebox{\baselineskip}[0in][0in]{\raisebox{#2}[0in][0in]{#3}}}}
     \def\rightbox#1{\makebox[0in][r]{#1}}
     \def\centbox#1{\makebox[0in]{#1}}
     \def\topbox#1{\raisebox{-\baselineskip}[0in][0in]{#1}}
     \def\midbox#1{\raisebox{-0.5\baselineskip}[0in][0in]{#1}}
\vspace{3cm}
\title{Matrix Theory (EE5609) Assignment 3}
\author{Arkadipta De \\MTech Artificial Intelligence\\Roll No - AI20MTECH14002}
\maketitle
\newpage
\bigskip
%\renewcommand{\thefigure}{\theenumi}
\renewcommand{\thetable}{\theenumi}
\begin{abstract}
This assignment proves that matrix multiplication is not commutative.
\end{abstract}
The code for this solution can be found from 
\begin{lstlisting}
https://github.com/Arko98/EE5609/blob/master/Assignment_3/Codes/Solution_3.py
\end{lstlisting}
\section{\textbf{Problem Statement}}Show that\\
\begin{align*}
\myvec{5 & -1\\6 & 7}\myvec{2 & 1\\3 & 4} \not= \myvec{2 & 1\\3 & 4}\myvec{5 & -1\\6 & 7}
\end{align*}

\section{\textbf{Solution}}
Let the two matrices be $\vec{A}$ = \myvec{5 & -1\\6 & 7} and $\vec{B}$ = \myvec{2 & 1\\3 & 4}. From the problem we have to prove the following
\begin{align}\label{eq1}
\vec{A}\vec{B} \not= \vec{B}\vec{A}
\end{align}
At first we compute left hand side of \ref{eq1}.
\begin{align}
\vec{A}\vec{B} &= \myvec{5 & -1\\6 & 7}\myvec{2 & 1\\3 & 4}\label{eqm1}\\
\implies \vec{A}\vec{B} &= \myvec{5\times2-1\times3 & 5\times1-1\times4\\6\times2+7\times3 & 6\times1+7\times4}\\
\implies\vec{A}\vec{B} &= \myvec{7 & 1\\33 & 34}\label{eq2}
\end{align}
Next, we compute right hand side of \ref{eq1}.
\begin{align}
\vec{B}\vec{A} &= \myvec{2 & 1\\3 & 4}\myvec{5 & -1\\6 & 7}\\
\implies \vec{B}\vec{A} &= \myvec{2\times5+1\times6 & 2\times(-1)+1\times7\\3\times5+4\times6 & 3\times(-1)+4\times7}\\
\implies\vec{B}\vec{A} &= \myvec{16 & 5\\39 & 25}\label{eq3}
\end{align}
Clearly we can see from equation \ref{eq2} and \ref{eq3} that the resultant matrices are not equal. Hence proved,
\begin{align*}
\myvec{5 & -1\\6 & 7}\myvec{2 & 1\\3 & 4} \not= \myvec{2 & 1\\3 & 4}\myvec{5 & -1\\6 & 7}
\end{align*}

\section{\textbf{Explanation}}
Matrix multiplication between two matrices $\vec{A}$ and $\vec{B}$ is the linear combination of the rows of matrix $\vec{B}$ using the elements of matrix $\vec{A}$. \\
If $\vec{R_B_1}$ and $\vec{R_B_2}$ are first row and second row of the matrix $\vec{B}$ and $\vec{R_A_B_1}$ and $\vec{R_A_B_2}$ are first row and second row of the matrix $\vec{A}\vec{B}$ from equation \ref{eqm1} then rows of $\vec{A}\vec{B}$ is given by
\begin{align}
\vec{R_A_B_1} &= 5\vec{R_B_1}-\vec{R_B_2}\label{eqAB1}\\
\vec{R_A_B_2} &= 6\vec{R_B_1}+7\vec{R_B_2}\label{eqAB2}
\end{align}
Similarly if $\vec{R_A_1}$ and $\vec{R_A_2}$ are first row and second row of the matrix $\vec{A}$ and $\vec{R_B_A_1}$ and $\vec{R_B_A_2}$ are first row and second row of the matrix $\vec{B}\vec{A}$ from equation \ref{eqm1} then rows of $\vec{B}\vec{A}$ is given by
\begin{align}
\vec{R_B_A_1} &= 2\vec{R_A_1}+\vec{R_A_2}\label{eqAB3}\\
\vec{R_B_A_2} &= 3\vec{R_A_1}+4\vec{R_A_2}\label{eqAB4}
\end{align}
Clearly we can see from equations \ref{eqAB1} and \ref{eqAB3} that $\vec{R_A_B_1}\not=\vec{R_B_A_1}$ and from equations \ref{eqAB2} and \ref{eqAB4} that $\vec{R_A_B_2}\not=\vec{R_B_A_2}$. Hence matrix multiplication is generally not commutative.\\
\begin{comment}
\textbf{Python Code: }The code for the solution can be found at \url{https://github.com/Arko98/EE5609/blob/master/Assignment_2/Codes/Figure.py}
\end{comment}
\end{document}