\documentclass[journal,12pt,twocolumn]{IEEEtran}

\usepackage{setspace}
\usepackage{gensymb}

\singlespacing


\usepackage[cmex10]{amsmath}

\usepackage{amsthm}
\usepackage{hyperref}
\usepackage{mathrsfs}
\usepackage{txfonts}
\usepackage{stfloats}
\usepackage{bm}
\usepackage{cite}
\usepackage{cases}
\usepackage{subfig}

\usepackage{longtable}
\usepackage{multirow}

\usepackage{enumitem}
\usepackage{mathtools}
\usepackage{steinmetz}
\usepackage{tikz}
\usepackage{circuitikz}
\usepackage{verbatim}
\usepackage{tfrupee}
\usepackage[breaklinks=true]{hyperref}

\usepackage{tkz-euclide}

\usetikzlibrary{calc,math}
\usepackage{listings}
    \usepackage{color}                                            %%
    \usepackage{array}                                            %%
    \usepackage{longtable}                                        %%
    \usepackage{calc}                                             %%
    \usepackage{multirow}                                         %%
    \usepackage{hhline}                                           %%
    \usepackage{ifthen}                                           %%
    \usepackage{lscape}     
\usepackage{multicol}
\usepackage{chngcntr}

\DeclareMathOperator*{\Res}{Res}

\renewcommand\thesection{\arabic{section}}
\renewcommand\thesubsection{\thesection.\arabic{subsection}}
\renewcommand\thesubsubsection{\thesubsection.\arabic{subsubsection}}

\renewcommand\thesectiondis{\arabic{section}}
\renewcommand\thesubsectiondis{\thesectiondis.\arabic{subsection}}
\renewcommand\thesubsubsectiondis{\thesubsectiondis.\arabic{subsubsection}}


\hyphenation{op-tical net-works semi-conduc-tor}
\def\inputGnumericTable{}                                 %%

\lstset{
%language=C,
frame=single, 
breaklines=true,
columns=fullflexible
}
\begin{document}


\newtheorem{theorem}{Theorem}[section]
\newtheorem{problem}{Problem}
\newtheorem{proposition}{Proposition}[section]
\newtheorem{lemma}{Lemma}[section]
\newtheorem{corollary}[theorem]{Corollary}
\newtheorem{example}{Example}[section]
\newtheorem{definition}[problem]{Definition}

\newcommand{\BEQA}{\begin{eqnarray}}
\newcommand{\EEQA}{\end{eqnarray}}
\newcommand{\define}{\stackrel{\triangle}{=}}
\bibliographystyle{IEEEtran}
\providecommand{\mbf}{\mathbf}
\providecommand{\pr}[1]{\ensuremath{\Pr\left(#1\right)}}
\providecommand{\qfunc}[1]{\ensuremath{Q\left(#1\right)}}
\providecommand{\sbrak}[1]{\ensuremath{{}\left[#1\right]}}
\providecommand{\lsbrak}[1]{\ensuremath{{}\left[#1\right.}}
\providecommand{\rsbrak}[1]{\ensuremath{{}\left.#1\right]}}
\providecommand{\brak}[1]{\ensuremath{\left(#1\right)}}
\providecommand{\lbrak}[1]{\ensuremath{\left(#1\right.}}
\providecommand{\rbrak}[1]{\ensuremath{\left.#1\right)}}
\providecommand{\cbrak}[1]{\ensuremath{\left\{#1\right\}}}
\providecommand{\lcbrak}[1]{\ensuremath{\left\{#1\right.}}
\providecommand{\rcbrak}[1]{\ensuremath{\left.#1\right\}}}
\theoremstyle{remark}
\newtheorem{rem}{Remark}
\newcommand{\sgn}{\mathop{\mathrm{sgn}}}
\providecommand{\abs}[1]{\left\vert#1\right\vert}
\providecommand{\res}[1]{\Res\displaylimits_{#1}} 
\providecommand{\norm}[1]{\left\lVert#1\right\rVert}
%\providecommand{\norm}[1]{\lVert#1\rVert}
\providecommand{\mtx}[1]{\mathbf{#1}}
\providecommand{\mean}[1]{E\left[ #1 \right]}
\providecommand{\fourier}{\overset{\mathcal{F}}{ \rightleftharpoons}}
%\providecommand{\hilbert}{\overset{\mathcal{H}}{ \rightleftharpoons}}
\providecommand{\system}{\overset{\mathcal{H}}{ \longleftrightarrow}}
	%\newcommand{\solution}[2]{\textbf{Solution:}{#1}}
\newcommand{\solution}{\noindent \textbf{Solution: }}
\newcommand{\cosec}{\,\text{cosec}\,}
\providecommand{\dec}[2]{\ensuremath{\overset{#1}{\underset{#2}{\gtrless}}}}
\newcommand{\myvec}[1]{\ensuremath{\begin{pmatrix}#1\end{pmatrix}}}
\newcommand{\mydet}[1]{\ensuremath{\begin{vmatrix}#1\end{vmatrix}}}
\numberwithin{equation}{subsection}
\makeatletter
\@addtoreset{figure}{problem}
\makeatother
\let\StandardTheFigure\thefigure
\let\vec\mathbf
\renewcommand{\thefigure}{\theproblem}
\def\putbox#1#2#3{\makebox[0in][l]{\makebox[#1][l]{}\raisebox{\baselineskip}[0in][0in]{\raisebox{#2}[0in][0in]{#3}}}}
     \def\rightbox#1{\makebox[0in][r]{#1}}
     \def\centbox#1{\makebox[0in]{#1}}
     \def\topbox#1{\raisebox{-\baselineskip}[0in][0in]{#1}}
     \def\midbox#1{\raisebox{-0.5\baselineskip}[0in][0in]{#1}}
\vspace{3cm}
\title{Matrix Theory (EE5609) Assignment 1}
\author{Arkadipta De \\MTech Artificial Intelligence\\Roll No - AI20MTECH14002}
\date{September 01, 2020}
\maketitle
\newpage
\bigskip
\renewcommand{\thefigure}{\theenumi}
\renewcommand{\thetable}{\theenumi}
\begin{abstract}
This assignment solves a problem on checking whether two lines are parallel or perpendicular.
\end{abstract}
Below is the link to python code solution of this problem 

\section{\textbf{Problem Statement}}
Show that the line through the points \begin{pmatrix} 1 \\ -1 \\ 2  \end{pmatrix} and \begin{pmatrix} 3 \\ 4 \\ -2  \end{pmatrix} is parallel to the line through the points \begin{pmatrix} 0 \\ 3 \\ 2  \end{pmatrix} and \begin{pmatrix} 3 \\ 5 \\ 6  \end{pmatrix}.

\section{\textbf{Theory}}
The direction vector $\vec{A}$ for a line through the points \begin{pmatrix} x_1 \\ y_1 \\ z_1  \end{pmatrix} and \begin{pmatrix} X_2 \\ y_2 \\ z_2  \end{pmatrix} is given by
\begin{align}\label{eq1}
\vec{A}=\begin{pmatrix} x_2-x_1 \\ y_2-y_1 \\ z_2-z_1  \end{pmatrix}
\end{align}
For two lines having direction vectors  $\vec{A}$ and  $\vec{B}$  respectively, they will be perpendicular if the scalar product of the two direction vector is 0, 
\begin{align}\label{eq2}
\vec{A}\cdot\vec{B} = 0
\end{align}
Where scalar product of two vectors, $\vec{A}$= \begin{pmatrix} x_1 \\ y_1 \\ z_1 \end{pmatrix} and $\vec{B}$= \begin{pmatrix} x_2 \\ y_2 \\ z_2 \end{pmatrix} is defined by 
\begin{align}
\vec{A}\cdot\vec{B} &= \vec{A^T}\vec{B}= \begin{pmatrix} x_1 & y_1 & z_1 \end{pmatrix}\begin{pmatrix} x_2 \\ y_2 \\ z_2 \end{pmatrix}=x_1x_2+y_1y_2+z_1z_2 
\end{align}
And the two lines will be parallel if the cross product of the two direction vector is 0,
\begin{equation}\label{eq5}
\vec{A}\times\vec{B} = \vec{0}
\end{equation}
\section{\textbf{Solution}}
From the theory, the direction vector for the line through the points \begin{pmatrix} 1 \\ -1 \\ 2  \end{pmatrix} and \begin{pmatrix} 3 \\ 4 \\ -2  \end{pmatrix} is $\vec{A}= \begin{pmatrix} 2 \\ 5 \\ -4  \end{pmatrix}$ (using equations \ref{eq1} and \ref{eq2}). Similarly, the direction vector for the line through the points \begin{pmatrix} 0 \\ 3 \\ 2  \end{pmatrix} and \begin{pmatrix} 3 \\ 5 \\ 6  \end{pmatrix} is $\vec{B}=\begin{pmatrix} 3 \\ 2 \\ 4 \end{pmatrix}$ (using equations \ref{eq1} and \ref{eq2}).\\
To check if the two lines are perpendicular, we perform scalar product of the two direction vectors $\vec{A}$ and $\vec{B}$ using equation \ref{eq1} and \ref{eq2} as follows
\begin{align}
\vec{A}\cdot\vec{B} &=  \vec{A^T}\vec{B}\\
\implies\vec{A}\cdot\vec{B} &= \begin{pmatrix} 2 & 5 & -4 \end{pmatrix}\begin{pmatrix} 3 \\ 2 \\ 4 \end{pmatrix}\\
\implies\vec{A}\cdot\vec{B} &= 6+10-16=0
\end{align}

Thus the direction vectors of the two lines satisfies the equation \ref{eq1}, hence proved that the lines are \textbf{perpendicular}. Hence they are not \textbf{parallel} with each other.\\
\\
\textbf{Python Code: }The python code for the above solution can be found at - {\url{https://github.com/Arko98/EE5609/blob/master/Assignment_1/Codes/Solution_1.py}}

\end{document}