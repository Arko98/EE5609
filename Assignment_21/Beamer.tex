\documentclass[journal,12pt,twocolumn]{IEEEtran}
%
\usepackage{setspace}
\usepackage{gensymb}
\usepackage{siunitx}
\usepackage{tkz-euclide} 
\usepackage{textcomp}
\usepackage{standalone}
\usetikzlibrary{calc}

%\doublespacing
\singlespacing

%\usepackage{graphicx}
%\usepackage{amssymb}
%\usepackage{relsize}
\usepackage[cmex10]{amsmath}
%\usepackage{amsthm}
%\interdisplaylinepenalty=2500
%\savesymbol{iint}
%\usepackage{txfonts}
%\restoresymbol{TXF}{iint}
%\usepackage{wasysym}
\usepackage{amsthm}
%\usepackage{iithtlc}
\usepackage{mathrsfs}
\usepackage{txfonts}
\usepackage{stfloats}
\usepackage{bm}
\usepackage{cite}
\usepackage{cases}
\usepackage{subfig}
%\usepackage{xtab}
\usepackage{longtable}
\usepackage{multirow}
%\usepackage{algorithm}
%\usepackage{algpseudocode}
\usepackage{enumitem}
\usepackage{mathtools}
\usepackage{steinmetz}
\usepackage{tikz}
\usepackage{circuitikz}
\usepackage{verbatim}
\usepackage{tfrupee}
\usepackage[breaklinks=true]{hyperref}
%\usepackage{stmaryrd}
\usepackage{tkz-euclide} % loads  TikZ and tkz-base
%\usetkzobj{all}
\usetikzlibrary{calc,math}
\usepackage{listings}
    \usepackage{color}                                            %%
    \usepackage{array}                                            %%
    \usepackage{longtable}                                        %%
    \usepackage{calc}                                             %%
    \usepackage{multirow}                                         %%
    \usepackage{hhline}                                           %%
    \usepackage{ifthen}                                           %%
  %optionally (for landscape tables embedded in another document): %%
    \usepackage{lscape}     
\usepackage{multicol}
\usepackage{chngcntr}
\usepackage{amsmath}
\usepackage{cleveref}
%\usepackage{enumerate}

%\usepackage{wasysym}
%\newcounter{MYtempeqncnt}
\DeclareMathOperator*{\Res}{Res}
%\renewcommand{\baselinestretch}{2}
\renewcommand\thesection{\arabic{section}}
\renewcommand\thesubsection{\thesection.\arabic{subsection}}
\renewcommand\thesubsubsection{\thesubsection.\arabic{subsubsection}}

\renewcommand\thesectiondis{\arabic{section}}
\renewcommand\thesubsectiondis{\thesectiondis.\arabic{subsection}}
\renewcommand\thesubsubsectiondis{\thesubsectiondis.\arabic{subsubsection}}

% correct bad hyphenation here
\hyphenation{op-tical net-works semi-conduc-tor}
\def\inputGnumericTable{}                                 %%

\lstset{
%language=C,
frame=single, 
breaklines=true,
columns=fullflexible
}
%\lstset{
%language=tex,
%frame=single, 
%breaklines=true
%}
\usepackage{graphicx}
\usepackage{pgfplots}

\begin{document}
%


\newtheorem{theorem}{Theorem}[section]
\newtheorem{problem}{Problem}
\newtheorem{proposition}{Proposition}[section]
\newtheorem{lemma}{Lemma}[section]
\newtheorem{corollary}[theorem]{Corollary}
\newtheorem{example}{Example}[section]
\newtheorem{definition}[problem]{Definition}
%\newtheorem{thm}{Theorem}[section] 
%\newtheorem{defn}[thm]{Definition}
%\newtheorem{algorithm}{Algorithm}[section]
%\newtheorem{cor}{Corollary}
\newcommand{\BEQA}{\begin{eqnarray}}
\newcommand{\EEQA}{\end{eqnarray}}
\newcommand{\define}{\stackrel{\triangle}{=}}
\bibliographystyle{IEEEtran}
%\bibliographystyle{ieeetr}
\providecommand{\mbf}{\mathbf}
\providecommand{\pr}[1]{\ensuremath{\Pr\left(#1\right)}}
\providecommand{\qfunc}[1]{\ensuremath{Q\left(#1\right)}}
\providecommand{\sbrak}[1]{\ensuremath{{}\left[#1\right]}}
\providecommand{\lsbrak}[1]{\ensuremath{{}\left[#1\right.}}
\providecommand{\rsbrak}[1]{\ensuremath{{}\left.#1\right]}}
\providecommand{\brak}[1]{\ensuremath{\left(#1\right)}}
\providecommand{\lbrak}[1]{\ensuremath{\left(#1\right.}}
\providecommand{\rbrak}[1]{\ensuremath{\left.#1\right)}}
\providecommand{\cbrak}[1]{\ensuremath{\left\{#1\right\}}}
\providecommand{\lcbrak}[1]{\ensuremath{\left\{#1\right.}}
\providecommand{\rcbrak}[1]{\ensuremath{\left.#1\right\}}}
\theoremstyle{remark}
\newtheorem{rem}{Remark}
\newcommand{\sgn}{\mathop{\mathrm{sgn}}}



\providecommand{\abs}[1]{\left\vert#1\right\vert}
\providecommand{\res}[1]{\Res\displaylimits_{#1}} 
\providecommand{\norm}[1]{\left\lVert#1\right\rVert}
%\providecommand{\norm}[1]{\lVert#1\rVert}
\providecommand{\mtx}[1]{\mathbf{#1}}
\providecommand{\mean}[1]{E\left[ #1 \right]}
\providecommand{\fourier}{\overset{\mathcal{F}}{ \rightleftharpoons}}
%\providecommand{\hilbert}{\overset{\mathcal{H}}{ \rightleftharpoons}}
\providecommand{\system}{\overset{\mathcal{H}}{ \longleftrightarrow}}
	%\newcommand{\solution}[2]{\textbf{Solution:}{#1}}
\newcommand{\solution}{\noindent \textbf{Solution: }}
\newcommand{\cosec}{\,\text{cosec}\,}
\providecommand{\dec}[2]{\ensuremath{\overset{#1}{\underset{#2}{\gtrless}}}}
\newcommand{\myvec}[1]{\ensuremath{\begin{pmatrix}#1\end{pmatrix}}}
\newcommand{\mydet}[1]{\ensuremath{\begin{vmatrix}#1\end{vmatrix}}}
%\numberwithin{equation}{section}
\numberwithin{equation}{subsection}
%\numberwithin{problem}{section}
%\numberwithin{definition}{section}
\makeatletter
\@addtoreset{figure}{problem}
\makeatother
\let\StandardTheFigure\thefigure
\let\vec\mathbf
%\renewcommand{\thefigure}{\theproblem.\arabic{figure}}
\renewcommand{\thefigure}{\theproblem}
%\setlist[enumerate,1]{before=\renewcommand\theequation{\theenumi.\arabic{equation}}
%\counterwithin{equation}{enumi}
%\renewcommand{\theequation}{\arabic{subsection}.\arabic{equation}}
\def\putbox#1#2#3{\makebox[0in][l]{\makebox[#1][l]{}\raisebox{\baselineskip}[0in][0in]{\raisebox{#2}[0in][0in]{#3}}}}
     \def\rightbox#1{\makebox[0in][r]{#1}}
     \def\centbox#1{\makebox[0in]{#1}}
     \def\topbox#1{\raisebox{-\baselineskip}[0in][0in]{#1}}
     \def\midbox#1{\raisebox{-0.5\baselineskip}[0in][0in]{#1}}
\vspace{3cm}
\title{Matrix Theory (EE5609) Assignment 21}
\author{Arkadipta De\\MTech Artificial Intelligence\\AI20MTECH14002}

\maketitle
\newpage
%\tableofcontents
\bigskip
\renewcommand{\thefigure}{\theenumi}
\renewcommand{\thetable}{\theenumi}

\begin{abstract}
This document solves a evaluates a polynomial of a given square matrix. 
\end{abstract}
All the codes for the figure in this document can be found at
\begin{lstlisting}
https://github.com/Arko98/EE5609/blob/master/Assignment_21
\end{lstlisting}
\section{\textbf{Problem}}
Let $\mathbb{F}$ be a subfield of the complex numbers and let $\vec{A}$ be the following $2 \times 2$ matrix over $\mathbb{F}$,
\begin{align*}
\vec{A} = \myvec{2&1\\-1&3}
\end{align*}
For the following polynomial $f$ over $\mathbb{F}$,
\begin{align*}
f = x^3 - 1
\end{align*}
compute $f(\vec{A})$
\section{\textbf{Solution}}
\subsection{Method 1 (Using Diagonalization of Matrix)}
We first find the eigen values of the $\vec{A}$. We get the characteristic equation of $\vec{A}$ as follows,
\begin{align}
\det{(\vec{A} - \lambda \vec{I})} &= 0\\
\implies\lambda^2 - 5\lambda+7 &=0\label{eqEigen}
\end{align}
From \eqref{eqEigen} we get the eigen values of $\vec{A}$ as follows,
\begin{align}
\lambda_1 &= \frac{1}{2}(5+i\sqrt{3})\label{eval1}\\ 
\lambda_2 &= \frac{1}{2}(5-i\sqrt{3})\label{eval2}
\end{align}
And corresponding eigen vectors are as follows,
\begin{align}
\vec{e_1} &= \myvec{\frac{1}{2}(1-i\sqrt{3})&1}\label{ev1}\\
\vec{e_2} &= \myvec{\frac{1}{2}(1+i\sqrt{3})&1}\label{ev2}
\end{align}
From the eigen values in \eqref{eval1},\eqref{eval2} and eigen vectors \eqref{ev1} and \eqref{ev2} we get the eigenvalue diagonalization of $\vec{A}$ as follows,
\begin{align}
&\vec{A} = \vec{P}\vec{\Lambda}\vec{P}^{-1}\label{diag}
\end{align}
Where,
\begin{align}
\vec{P} &= \myvec{\frac{1}{2}(1-i\sqrt{3})&\frac{1}{2}(1+i\sqrt{3})\\1&1}\\
\vec{\Lambda} &= \myvec{\frac{1}{2}(5+i\sqrt{3})&0\\0&\frac{1}{2}(5-i\sqrt{3})}\\
\vec{P}^{-1} &= \myvec{-\frac{i}{\sqrt{3}}&\frac{1}{6}(3+i\sqrt{3})\\\frac{i}{\sqrt{3}}&\frac{1}{6}(3-i\sqrt{3})}
\end{align}
Hence,
\begin{align}
\vec{A}^3 &= \vec{P}\vec{\Lambda}^3\vec{P}^{-1}\\
\implies \vec{A}^3 &= \myvec{0&18\\-18&18}\label{Powerval}
\end{align}
Using \eqref{Powerval} in $f(\vec{A})$ we get,
\begin{align}
f(\vec{A}) &= \vec{A}^3 - \vec{I} \quad{\text{[Where $\vec{I}$ is $2 \times 2$ Identity matrix]}}\\
&= \myvec{1&18\\-18&19} - \myvec{1&0\\0&1}\\
&= \myvec{0&18\\-18&18}\label{Ans1}
\end{align}
Here, \eqref{Ans1} is the required answer.
\subsection{Method 2 (Direct Method)}
From the equation of polynomial we get,
\begin{align}
f(\vec{A}) &= \vec{A}^3 - \vec{I} \quad{\text{[Where $\vec{I}$ is $2 \times 2$ Identity matrix]}}\\
&= \myvec{2&1\\-1&3}^3 - \myvec{1&0\\0&1}\\
&= \myvec{2&1\\-1&3}\myvec{2&1\\-1&3}\myvec{2&1\\-1&3} - \myvec{1&0\\0&1}\\
&= \myvec{1&18\\-18&19} - \myvec{1&0\\0&1}\\ 
&= \myvec{0&18\\-18&18}\label{Ans2}
\end{align}
Here, \eqref{Ans2} is the required answer.
\end{document}