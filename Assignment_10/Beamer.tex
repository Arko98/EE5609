\documentclass[journal,12pt,twocolumn]{IEEEtran}
%
\usepackage{setspace}
\usepackage{gensymb}
\usepackage{siunitx}
\usepackage{tkz-euclide} 
\usepackage{textcomp}
\usepackage{standalone}
\usetikzlibrary{calc}

%\doublespacing
\singlespacing

%\usepackage{graphicx}
%\usepackage{amssymb}
%\usepackage{relsize}
\usepackage[cmex10]{amsmath}
%\usepackage{amsthm}
%\interdisplaylinepenalty=2500
%\savesymbol{iint}
%\usepackage{txfonts}
%\restoresymbol{TXF}{iint}
%\usepackage{wasysym}
\usepackage{amsthm}
%\usepackage{iithtlc}
\usepackage{mathrsfs}
\usepackage{txfonts}
\usepackage{stfloats}
\usepackage{bm}
\usepackage{cite}
\usepackage{cases}
\usepackage{subfig}
%\usepackage{xtab}
\usepackage{longtable}
\usepackage{multirow}
%\usepackage{algorithm}
%\usepackage{algpseudocode}
\usepackage{enumitem}
\usepackage{mathtools}
\usepackage{steinmetz}
\usepackage{tikz}
\usepackage{circuitikz}
\usepackage{verbatim}
\usepackage{tfrupee}
\usepackage[breaklinks=true]{hyperref}
%\usepackage{stmaryrd}
\usepackage{tkz-euclide} % loads  TikZ and tkz-base
%\usetkzobj{all}
\usetikzlibrary{calc,math}
\usepackage{listings}
    \usepackage{color}                                            %%
    \usepackage{array}                                            %%
    \usepackage{longtable}                                        %%
    \usepackage{calc}                                             %%
    \usepackage{multirow}                                         %%
    \usepackage{hhline}                                           %%
    \usepackage{ifthen}                                           %%
  %optionally (for landscape tables embedded in another document): %%
    \usepackage{lscape}     
\usepackage{multicol}
\usepackage{chngcntr}
\usepackage{amsmath}
\usepackage{cleveref}
%\usepackage{enumerate}

%\usepackage{wasysym}
%\newcounter{MYtempeqncnt}
\DeclareMathOperator*{\Res}{Res}
%\renewcommand{\baselinestretch}{2}
\renewcommand\thesection{\arabic{section}}
\renewcommand\thesubsection{\thesection.\arabic{subsection}}
\renewcommand\thesubsubsection{\thesubsection.\arabic{subsubsection}}

\renewcommand\thesectiondis{\arabic{section}}
\renewcommand\thesubsectiondis{\thesectiondis.\arabic{subsection}}
\renewcommand\thesubsubsectiondis{\thesubsectiondis.\arabic{subsubsection}}

% correct bad hyphenation here
\hyphenation{op-tical net-works semi-conduc-tor}
\def\inputGnumericTable{}                                 %%

\lstset{
%language=C,
frame=single, 
breaklines=true,
columns=fullflexible
}
%\lstset{
%language=tex,
%frame=single, 
%breaklines=true
%}
\usepackage{graphicx}
\usepackage{pgfplots}

\begin{document}
%


\newtheorem{theorem}{Theorem}[section]
\newtheorem{problem}{Problem}
\newtheorem{proposition}{Proposition}[section]
\newtheorem{lemma}{Lemma}[section]
\newtheorem{corollary}[theorem]{Corollary}
\newtheorem{example}{Example}[section]
\newtheorem{definition}[problem]{Definition}
%\newtheorem{thm}{Theorem}[section] 
%\newtheorem{defn}[thm]{Definition}
%\newtheorem{algorithm}{Algorithm}[section]
%\newtheorem{cor}{Corollary}
\newcommand{\BEQA}{\begin{eqnarray}}
\newcommand{\EEQA}{\end{eqnarray}}
\newcommand{\define}{\stackrel{\triangle}{=}}
\bibliographystyle{IEEEtran}
%\bibliographystyle{ieeetr}
\providecommand{\mbf}{\mathbf}
\providecommand{\pr}[1]{\ensuremath{\Pr\left(#1\right)}}
\providecommand{\qfunc}[1]{\ensuremath{Q\left(#1\right)}}
\providecommand{\sbrak}[1]{\ensuremath{{}\left[#1\right]}}
\providecommand{\lsbrak}[1]{\ensuremath{{}\left[#1\right.}}
\providecommand{\rsbrak}[1]{\ensuremath{{}\left.#1\right]}}
\providecommand{\brak}[1]{\ensuremath{\left(#1\right)}}
\providecommand{\lbrak}[1]{\ensuremath{\left(#1\right.}}
\providecommand{\rbrak}[1]{\ensuremath{\left.#1\right)}}
\providecommand{\cbrak}[1]{\ensuremath{\left\{#1\right\}}}
\providecommand{\lcbrak}[1]{\ensuremath{\left\{#1\right.}}
\providecommand{\rcbrak}[1]{\ensuremath{\left.#1\right\}}}
\theoremstyle{remark}
\newtheorem{rem}{Remark}
\newcommand{\sgn}{\mathop{\mathrm{sgn}}}
\providecommand{\abs}[1]{\left\vert#1\right\vert}
\providecommand{\res}[1]{\Res\displaylimits_{#1}} 
\providecommand{\norm}[1]{\left\lVert#1\right\rVert}
%\providecommand{\norm}[1]{\lVert#1\rVert}
\providecommand{\mtx}[1]{\mathbf{#1}}
\providecommand{\mean}[1]{E\left[ #1 \right]}
\providecommand{\fourier}{\overset{\mathcal{F}}{ \rightleftharpoons}}
%\providecommand{\hilbert}{\overset{\mathcal{H}}{ \rightleftharpoons}}
\providecommand{\system}{\overset{\mathcal{H}}{ \longleftrightarrow}}
	%\newcommand{\solution}[2]{\textbf{Solution:}{#1}}
\newcommand{\solution}{\noindent \textbf{Solution: }}
\newcommand{\cosec}{\,\text{cosec}\,}
\providecommand{\dec}[2]{\ensuremath{\overset{#1}{\underset{#2}{\gtrless}}}}
\newcommand{\myvec}[1]{\ensuremath{\begin{pmatrix}#1\end{pmatrix}}}
\newcommand{\mydet}[1]{\ensuremath{\begin{vmatrix}#1\end{vmatrix}}}
%\numberwithin{equation}{section}
\numberwithin{equation}{subsection}
%\numberwithin{problem}{section}
%\numberwithin{definition}{section}
\makeatletter
\@addtoreset{figure}{problem}
\makeatother
\let\StandardTheFigure\thefigure
\let\vec\mathbf
%\renewcommand{\thefigure}{\theproblem.\arabic{figure}}
\renewcommand{\thefigure}{\theproblem}
%\setlist[enumerate,1]{before=\renewcommand\theequation{\theenumi.\arabic{equation}}
%\counterwithin{equation}{enumi}
%\renewcommand{\theequation}{\arabic{subsection}.\arabic{equation}}
\def\putbox#1#2#3{\makebox[0in][l]{\makebox[#1][l]{}\raisebox{\baselineskip}[0in][0in]{\raisebox{#2}[0in][0in]{#3}}}}
     \def\rightbox#1{\makebox[0in][r]{#1}}
     \def\centbox#1{\makebox[0in]{#1}}
     \def\topbox#1{\raisebox{-\baselineskip}[0in][0in]{#1}}
     \def\midbox#1{\raisebox{-0.5\baselineskip}[0in][0in]{#1}}
\vspace{3cm}
\title{Matrix Theory (EE5609) Assignment 10}
\author{Arkadipta De\\MTech Artificial Intelligence\\AI20MTECH14002}

\maketitle
\newpage
%\tableofcontents
\bigskip
\renewcommand{\thefigure}{\theenumi}
\renewcommand{\thetable}{\theenumi}

\begin{abstract}
This finds the distance between a given point and a plane using Singular Value Decomposition.
\end{abstract}

All the codes for the figure in this document can be found at
\begin{lstlisting}
https://github.com/Arko98/EE5609/blob/master/Assignment_10
\end{lstlisting}

\section{\textbf{Problem}}
Find the distance of the point \myvec{2\\5\\-3} from the plane \myvec{6&-3&2}$\vec{x}$ = 4
 \section{\textbf{Solution}}
First we find orthogonal vectors $\vec{m_1}$ and $\vec{m_2}$ to the given normal vector $\vec{n}$. Let, \vec{m} = $\myvec{a\\b\\c}$, then
\begin{align}
\vec{m^T}\vec{n} &= 0\\
\implies\myvec{a,b,c}\myvec{6\\-3\\2} &= 0\\
\implies6a-3b+2c &= 0\\
\intertext{Putting a=1 and b=0 we get,}
\vec{m_1} &= \myvec{1\\0\\3}\\
\intertext{Putting a=0 and b=1 we get,}
\vec{m_2} &= \myvec{0\\1\\\frac{3}{2}}
\end{align}
Now we solve the equation,
\begin{align}
\vec{M}\vec{x} &= \vec{b}\label{eq1}\\
\intertext{Putting values in \eqref{eq1},}
\myvec{1&0\\0&1\\3&\frac{3}{2}}\vec{x} &= \myvec{2\\5\\-3}\label{eq2}
\end{align}
Now, to solve \eqref{eq2}, we perform Singular Value Decomposition on $\vec{M}$ as follows,
\begin{align}
\vec{M}=\vec{U}\vec{S}\vec{V}^T\label{eqSVD}
\end{align}
Where the columns of $\vec{V}$ are the eigen vectors of $\vec{M}^T\vec{M}$ ,the columns of $\vec{U}$ are the eigen vectors of $\vec{M}\vec{M}^T$ and $\vec{S}$ is diagonal matrix of singular value of eigenvalues of $\vec{M}^T\vec{M}$.
\begin{align}
\vec{M}^T\vec{M}=\myvec{10&\frac{9}{2}\\\frac{9}{2}&\frac{13}{4}}\label{eqMTM}\\
\vec{M}\vec{M}^T=\myvec{1&0&3\\0&1&\frac{3}{2}\\3&\frac{3}{2}&\frac{45}{4}}
\end{align}
From \eqref{eq1} putting \eqref{eqSVD} we get,
\begin{align}
\vec{U}\vec{S}\vec{V}^T\vec{x} & = \vec{b}\\
\implies\vec{x} &= \vec{V}\vec{S_+}\vec{U^T}\vec{b}\label{eqX}\label{eqX}
\end{align}
Where $\vec{S_+}$ is Moore-Penrose Pseudo-Inverse of $\vec{S}$.Now, calculating eigen value of $\vec{M}\vec{M}^T$,
\begin{align}
\mydet{\vec{M}\vec{M}^T - \lambda\vec{I}} &= 0\\
\implies\myvec{1-\lambda&0&3\\0&1-\lambda&\frac{3}{2}\\3&\frac{3}{2}&\frac{45}{4}-\lambda} &=0\\
\implies\lambda^3-\frac{53}{4}\lambda^2+\frac{49}{4}\lambda &=0
\end{align}
Hence eigen values of $\vec{M}\vec{M}^T$ are,
\begin{align}
\lambda_1 &= \frac{49}{4}\\
\lambda_2 &= 1\\
\lambda_3 &= 0
\end{align}
Hence the eigen vectors of $\vec{M}\vec{M}^T$ are,
\begin{align}
\vec{u}_1=\myvec{\frac{4}{15}\\\frac{2}{15}\\1},
\vec{u}_2=\myvec{-\frac{1}{2}\\1\\0},
\vec{u}_3=\myvec{-3\\-\frac{3}{2}\\1}
\intertext{Normalizing the eigen vectors we get,}
\vec{u}_1=\myvec{\frac{4}{7\sqrt{5}}\\\frac{2}{7\sqrt{5}}\\\frac{3\sqrt{5}}{7}},
\vec{u}_2=\myvec{-\frac{1}{\sqrt{5}}\\\frac{2}{\sqrt{5}}\\0},
\vec{u}_3=\myvec{-\frac{6}{7}\\-\frac{3}{7}\\\frac{2}{7}}
\end{align}
Hence we obtain $\vec{U}$ of \eqref{eqSVD} as follows,
\begin{align}
\vec{U}=\myvec{\frac{4}{7\sqrt{5}}&-\frac{1}{\sqrt{5}}&-\frac{6}{7}\\\frac{2}{7\sqrt{5}}&\frac{2}{\sqrt{5}}&-\frac{3}{7}\\\frac{3\sqrt{5}}{7}&0&\frac{2}{7}}\label{eqU}
\end{align}
After computing the singular values from eigen values $\lambda_1, \lambda_2, \lambda_3$ we get $\vec{S}$ of \eqref{eqSVD} as follows,
\begin{align}
\vec{S}=\myvec{\frac{7}{2}&0\\0&1\\0&0}\label{eqS}
\end{align}
Now, calculating eigen value of $\vec{M}^T\vec{M}$,
\begin{align}
\mydet{\vec{M}^T\vec{M} - \lambda\vec{I}} &= 0\\
\implies\myvec{10-\lambda&\frac{9}{2}\\\frac{9}{2}&\frac{13}{4}-\lambda} &=0\\
\implies\lambda^2-\frac{53}{4}\lambda+\frac{49}{4} &=0
\end{align}
Hence eigen values of $\vec{M}^T\vec{M}$ are,
\begin{align}
\lambda_4 &= \frac{49}{4}\\
\lambda_5 &= 1
\end{align}
Hence the eigen vectors of $\vec{M}^T\vec{M}$ are,
\begin{align}
\vec{v}_1=\myvec{2\\1},
\vec{v}_2=\myvec{-\frac{1}{2}\\1}
\intertext{Normalizing the eigen vectors we get,}
\vec{v}_1=\myvec{\frac{2}{\sqrt{5}}\\\frac{1}{\sqrt{5}}},
\vec{v}_2=\myvec{-\frac{1}{\sqrt{5}}\\\frac{2}{\sqrt{5}}}
\end{align}
Hence we obtain $\vec{V}$ of \eqref{eqSVD} as follows,
\begin{align}
\vec{V}=\myvec{\frac{2}{\sqrt{5}}&-\frac{1}{\sqrt{5}}\\\frac{1}{\sqrt{5}}&\frac{2}{\sqrt{5}}}
\end{align}
Finally from \eqref{eqSVD} we get the Singualr Value Decomposition of $\vec{M}$ as follows,
\begin{align}
\vec{M} = \myvec{\frac{4}{7\sqrt{5}}&-\frac{1}{\sqrt{5}}&-\frac{6}{7}\\\frac{2}{7\sqrt{5}}&\frac{2}{\sqrt{5}}&-\frac{3}{7}\\\frac{3\sqrt{5}}{7}&0&\frac{2}{7}}\myvec{\frac{7}{2}&0\\0&1\\0&0}\myvec{\frac{2}{\sqrt{5}}&-\frac{1}{\sqrt{5}}\\\frac{1}{\sqrt{5}}&\frac{2}{\sqrt{5}}}^T
\end{align}
Now, Moore-Penrose Pseudo inverse of $\vec{S}$ is given by,
\begin{align}
\vec{S_+} = \myvec{\frac{2}{7}&0&0\\0&1&0}
\end{align}
From \eqref{eqX} we get,
\begin{align}
\vec{U}^T\vec{b}&=\myvec{-\frac{27}{7\sqrt{5}}\\\frac{8}{7\sqrt{5}}\\-\frac{33}{7}}\\
\vec{S_+}\vec{U}^T\vec{b}&=\myvec{-\frac{54}{49\sqrt{5}}\\\frac{8}{7\sqrt{5}}}\\
\vec{x} = \vec{V}\vec{S_+}\vec{U}^T\vec{b} &= \myvec{-\frac{100}{49}\\\frac{146}{49}}\label{eqXSol1}
\end{align}
Verifying the solution of \eqref{eqXSol1} using,
\begin{align}
\vec{M}^T\vec{M}\vec{x} = \vec{M}^T\vec{b}\label{eqVerify}
\end{align}
Evaluating the R.H.S in \eqref{eqVerify} we get,
\begin{align}
\vec{M}^T\vec{M}\vec{x} &= \myvec{-7\\\frac{1}{2}}\\
\implies\myvec{10&\frac{9}{2}\\\frac{9}{2}&\frac{13}{4}}\vec{x} &= \myvec{-7\\\frac{1}{2}}\label{eqMateq}
\end{align}
Solving the augmented matrix of \eqref{eqMateq} we get,
\begin{align}
\myvec{10&\frac{9}{2}&-7\\\frac{9}{2}&\frac{13}{4}&\frac{1}{2}} &\xleftrightarrow{R_1=\frac{1}{10}R_1}\myvec{1&\frac{9}{20}&-\frac{7}{10}\\\frac{9}{2}&\frac{13}{4}&\frac{1}{2}}\\
&\xleftrightarrow{R_2=R_2-\frac{9}{2}R_1}\myvec{1&\frac{9}{20}&-\frac{7}{10}\\0&\frac{49}{40}&\frac{73}{20}}\\
&\xleftrightarrow{R_2=\frac{40}{49}R_2}\myvec{1&\frac{9}{20}&-\frac{7}{10}\\0&1&\frac{146}{49}}\\
&\xleftrightarrow{R_1=R_1-\frac{9}{20}R_1}\myvec{1&0&-\frac{100}{49}\\0&1&\frac{146}{49}}
\end{align}
Hence, Solution of \eqref{eqVerify} is given by,
\begin{align}
\vec{x}=\myvec{-\frac{100}{49}\\\frac{146}{49}}\label{eqX2}
\end{align}
Comparing results of $\vec{x}$ from \eqref{eqXSol1} and \eqref{eqX2} we conclude that the solution is verified.
 \end{document}