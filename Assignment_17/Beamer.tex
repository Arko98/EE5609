\documentclass[journal,12pt,twocolumn]{IEEEtran}
%
\usepackage{setspace}
\usepackage{gensymb}
\usepackage{siunitx}
\usepackage{tkz-euclide} 
\usepackage{textcomp}
\usepackage{standalone}
\usetikzlibrary{calc}

%\doublespacing
\singlespacing

%\usepackage{graphicx}
%\usepackage{amssymb}
%\usepackage{relsize}
\usepackage[cmex10]{amsmath}
%\usepackage{amsthm}
%\interdisplaylinepenalty=2500
%\savesymbol{iint}
%\usepackage{txfonts}
%\restoresymbol{TXF}{iint}
%\usepackage{wasysym}
\usepackage{amsthm}
%\usepackage{iithtlc}
\usepackage{mathrsfs}
\usepackage{txfonts}
\usepackage{stfloats}
\usepackage{bm}
\usepackage{cite}
\usepackage{cases}
\usepackage{subfig}
%\usepackage{xtab}
\usepackage{longtable}
\usepackage{multirow}
%\usepackage{algorithm}
%\usepackage{algpseudocode}
\usepackage{enumitem}
\usepackage{mathtools}
\usepackage{steinmetz}
\usepackage{tikz}
\usepackage{circuitikz}
\usepackage{verbatim}
\usepackage{tfrupee}
\usepackage[breaklinks=true]{hyperref}
%\usepackage{stmaryrd}
\usepackage{tkz-euclide} % loads  TikZ and tkz-base
%\usetkzobj{all}
\usetikzlibrary{calc,math}
\usepackage{listings}
    \usepackage{color}                                            %%
    \usepackage{array}                                            %%
    \usepackage{longtable}                                        %%
    \usepackage{calc}                                             %%
    \usepackage{multirow}                                         %%
    \usepackage{hhline}                                           %%
    \usepackage{ifthen}                                           %%
  %optionally (for landscape tables embedded in another document): %%
    \usepackage{lscape}     
\usepackage{multicol}
\usepackage{chngcntr}
\usepackage{amsmath}
\usepackage{cleveref}
%\usepackage{enumerate}

%\usepackage{wasysym}
%\newcounter{MYtempeqncnt}
\DeclareMathOperator*{\Res}{Res}
%\renewcommand{\baselinestretch}{2}
\renewcommand\thesection{\arabic{section}}
\renewcommand\thesubsection{\thesection.\arabic{subsection}}
\renewcommand\thesubsubsection{\thesubsection.\arabic{subsubsection}}

\renewcommand\thesectiondis{\arabic{section}}
\renewcommand\thesubsectiondis{\thesectiondis.\arabic{subsection}}
\renewcommand\thesubsubsectiondis{\thesubsectiondis.\arabic{subsubsection}}

% correct bad hyphenation here
\hyphenation{op-tical net-works semi-conduc-tor}
\def\inputGnumericTable{}                                 %%

\lstset{
%language=C,
frame=single, 
breaklines=true,
columns=fullflexible
}
%\lstset{
%language=tex,
%frame=single, 
%breaklines=true
%}
\usepackage{graphicx}
\usepackage{pgfplots}

\begin{document}
%


\newtheorem{theorem}{Theorem}[section]
\newtheorem{problem}{Problem}
\newtheorem{proposition}{Proposition}[section]
\newtheorem{lemma}{Lemma}[section]
\newtheorem{corollary}[theorem]{Corollary}
\newtheorem{example}{Example}[section]
\newtheorem{definition}[problem]{Definition}
%\newtheorem{thm}{Theorem}[section] 
%\newtheorem{defn}[thm]{Definition}
%\newtheorem{algorithm}{Algorithm}[section]
%\newtheorem{cor}{Corollary}
\newcommand{\BEQA}{\begin{eqnarray}}
\newcommand{\EEQA}{\end{eqnarray}}
\newcommand{\define}{\stackrel{\triangle}{=}}
\bibliographystyle{IEEEtran}
%\bibliographystyle{ieeetr}
\providecommand{\mbf}{\mathbf}
\providecommand{\pr}[1]{\ensuremath{\Pr\left(#1\right)}}
\providecommand{\qfunc}[1]{\ensuremath{Q\left(#1\right)}}
\providecommand{\sbrak}[1]{\ensuremath{{}\left[#1\right]}}
\providecommand{\lsbrak}[1]{\ensuremath{{}\left[#1\right.}}
\providecommand{\rsbrak}[1]{\ensuremath{{}\left.#1\right]}}
\providecommand{\brak}[1]{\ensuremath{\left(#1\right)}}
\providecommand{\lbrak}[1]{\ensuremath{\left(#1\right.}}
\providecommand{\rbrak}[1]{\ensuremath{\left.#1\right)}}
\providecommand{\cbrak}[1]{\ensuremath{\left\{#1\right\}}}
\providecommand{\lcbrak}[1]{\ensuremath{\left\{#1\right.}}
\providecommand{\rcbrak}[1]{\ensuremath{\left.#1\right\}}}
\theoremstyle{remark}
\newtheorem{rem}{Remark}
\newcommand{\sgn}{\mathop{\mathrm{sgn}}}



\providecommand{\abs}[1]{\left\vert#1\right\vert}
\providecommand{\res}[1]{\Res\displaylimits_{#1}} 
\providecommand{\norm}[1]{\left\lVert#1\right\rVert}
%\providecommand{\norm}[1]{\lVert#1\rVert}
\providecommand{\mtx}[1]{\mathbf{#1}}
\providecommand{\mean}[1]{E\left[ #1 \right]}
\providecommand{\fourier}{\overset{\mathcal{F}}{ \rightleftharpoons}}
%\providecommand{\hilbert}{\overset{\mathcal{H}}{ \rightleftharpoons}}
\providecommand{\system}{\overset{\mathcal{H}}{ \longleftrightarrow}}
	%\newcommand{\solution}[2]{\textbf{Solution:}{#1}}
\newcommand{\solution}{\noindent \textbf{Solution: }}
\newcommand{\cosec}{\,\text{cosec}\,}
\providecommand{\dec}[2]{\ensuremath{\overset{#1}{\underset{#2}{\gtrless}}}}
\newcommand{\myvec}[1]{\ensuremath{\begin{pmatrix}#1\end{pmatrix}}}
\newcommand{\mydet}[1]{\ensuremath{\begin{vmatrix}#1\end{vmatrix}}}
%\numberwithin{equation}{section}
\numberwithin{equation}{subsection}
%\numberwithin{problem}{section}
%\numberwithin{definition}{section}
\makeatletter
\@addtoreset{figure}{problem}
\makeatother
\let\StandardTheFigure\thefigure
\let\vec\mathbf
%\renewcommand{\thefigure}{\theproblem.\arabic{figure}}
\renewcommand{\thefigure}{\theproblem}
%\setlist[enumerate,1]{before=\renewcommand\theequation{\theenumi.\arabic{equation}}
%\counterwithin{equation}{enumi}
%\renewcommand{\theequation}{\arabic{subsection}.\arabic{equation}}
\def\putbox#1#2#3{\makebox[0in][l]{\makebox[#1][l]{}\raisebox{\baselineskip}[0in][0in]{\raisebox{#2}[0in][0in]{#3}}}}
     \def\rightbox#1{\makebox[0in][r]{#1}}
     \def\centbox#1{\makebox[0in]{#1}}
     \def\topbox#1{\raisebox{-\baselineskip}[0in][0in]{#1}}
     \def\midbox#1{\raisebox{-0.5\baselineskip}[0in][0in]{#1}}
\vspace{3cm}
\title{Matrix Theory (EE5609) Assignment 17}
\author{Arkadipta De\\MTech Artificial Intelligence\\AI20MTECH14002}

\maketitle
\newpage
%\tableofcontents
\bigskip
\renewcommand{\thefigure}{\theenumi}
\renewcommand{\thetable}{\theenumi}

\begin{abstract}
This document proves the invertibility of a certain linear operator. 
\end{abstract}
All the codes for the figure in this document can be found at
\begin{lstlisting}
https://github.com/Arko98/EE5609/blob/master/Assignment_17
\end{lstlisting}
\section{\textbf{Problem}}
Let $T$ be a linear operator on the finite-dimensional space $\mathbb{V}$. Suppose there is a linear operator $U$ on $\mathbb{V}$ such that $TU = I$. Prove that $T$ is invertible and $U = T^{-1}$. Give an example which shows that this is false when $\mathbb{V}$ is not finite-dimensional.
\section{\textbf{Solution}}
\subsection{Proof}
Let $T:\mathbb{V} \xrightarrow{} \mathbb{V}$ be a linear operator, where $\mathbb{V}$ is a finite dimensional vectors space and $U:\mathbb{V} \xrightarrow{} \mathbb{V}$ is also a linear operator such that,
\begin{align}
TU &= I
\intertext{Where, $I$ is an identity transformation. Now we know that linear transformations are functions. Hence,}
TU &= I \quad{\text{is a function}}\\
\implies I &:\mathbb{V}\xrightarrow{}\mathbb{V}\\
\intertext{Such that $T(V) = V$. Defining $TU &:\mathbb{V}\xrightarrow{}\mathbb{V}$ to be a linear operator, we have}
T[U(V_i)] &= V_i \qquad{\text{[$V_i \in \mathbb{V}$]}}
\end{align}


\begin{comment}
Let $\vec{V_1},\vec{V_2} \in \mathbb{V}$ then,
\begin{align}
\intertext{If $\vec{V_1} \ne \vec{V_2}$ then, $T[U(\vec{V_1})] \ne T[U(\vec{V_2})]$. Hence,}
T \quad{\text{must be one-one function}}\label{oneone}
\intertext{Again, $T$ is linear operator on finite dimensional vector space. Hence,}
T \quad{\text{must be onto function}}\label{onto}
\intertext{From \eqref{oneone} and \eqref{onto} we get,}
T \quad{\text{is invertible function}}\label{inv}
\end{align}
\end{comment}
Now we show in the below Table that $T$ is one-one and onto as follows,\\
\setcounter{table}{0} \renewcommand{\thetable}{\arabic{table}}
\begin{table}[h!]
\centering\renewcommand\cellalign{lc}
\makegapedcells
\begin{tabular}{|c|c|} \hline
\textbf{Proof} & \textbf{Conclusion}  \\ \hline
\makecell{Let $\vec{V_1},\vec{V_2} \in \mathbb{V}$ then, &
         \\If $\vec{V_1} \ne \vec{V_2}$ then, & $T$ is one-one function
         \\$T[U(\vec{V_1})] \ne T[U(\vec{V_2})]$} &  \\ \hline
\makecell{$T$ is linear operator on & 
          \\finite dimensional & $T$ is onto function 
          \\ vector space} & \\\hline
\end{tabular}
\caption{Proof of Invertibility of transformation}
\label{tab:invert}
\end{table}\\
\\Hence we get from Table \ref{tab:invert} that, $T$ is invertible.
Hence we get the following,
\begin{align}
TT^{-1} &= I\\
\intertext{Where $T^{-1}$ is an inverse function of linear operator $T$. Hence,}
TT^{-1} &= I = TU\\
\implies T^{-1}(TT^{-1}) &= T^{-1}(TU)\\
\implies T^{-1}(I) &= IU\\
\implies T^{-1} &= U\label{proof}
\end{align}
Hence from \eqref{proof} it is proven that $T$ is invertible and $T^{-1} = U$
\subsection{Example}
Let $D$ be the differential operator $D:\mathbb{V} \xrightarrow{} \mathbb{V}$ where $\mathbb{V}$ is a space of polynomial functions in one variable over $\mathbb{R}$. Hence,
\begin{align}
D(c_0+c_1x+\dots+c_nx^n) &= c_1+c^{\prime}_2x+\dots+c^{\prime}_nx^{n-1}
\end{align}
And, $U:\mathbb{V} \xrightarrow{} \mathbb{V}$ is another linear operator such that,
\begin{align}
U(c_0+c_1x+\dots+c_nx^n) &= c_0x+c_1\frac{x^2}{2}+\dots+c_n\frac{x^{n+1}}{n+1}
\end{align}
Then $UD:\mathbb{V} \xrightarrow{} \mathbb{V}$ is a linear operator such that,
\begin{align}
&UD(c_0+c_1x+\dots+c_nx^n) \\
&= U[D(c_0x+c_1\frac{x^2}{2}+\dots+c_n\frac{x^{n+1}}{n+1})]\\
&= U[c_1+c^{\prime}_2x+\dots+c^{\prime}_nx^{n-1}]\\
&= c_1x+c_2\frac{x^2}{2}+\dots+c_n\frac{x^{n}}{n}\label{eq1}\
\end{align}
Hence, from \eqref{eq1},
\begin{align}
UD \ne I\label{show1}
\end{align}
Again, $DU:\mathbb{V} \xrightarrow{} \mathbb{V}$ is a linear operator such that,
\begin{align}
&DU(c_0+c_1x+\dots+c_nx^n) \\
&= D[U(c_0x+c_1\frac{x^2}{2}+\dots+c_n\frac{x^{n+1}}{n+1})]\\
&= D[c_0x+c_1\frac{x^2}{2}+\dots+c_n\frac{x^{n+1}}{n+1}]\\
&= c_0+c_1\frac{2x}{2}+\dots+c_n\frac{(n+1)x^{n}}{n+1}\\
&= c_0+c_1x+\dots+c_nx^n\label{eq2}
\end{align}
Hence, from \eqref{eq2},
\begin{align}
DU = I\label{show2}
\end{align}
Hence, from \eqref{show1} and \eqref{show2}, $D$ is not invertible.
\end{document}
